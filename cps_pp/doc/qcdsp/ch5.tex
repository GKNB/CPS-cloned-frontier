\chapter{I/O on the QCDSP}
\section{Introduction}
In this chapter we discuss the facilities afforded by QCDSP to perform
input and output (I/O). We consider three aspects: parallel I/O. 
Serial File I/O and I/O to the terminal screen during job runtime.

\section{Parallel I/O}
The QCDSP is a massively parallel supercomputer. Hence there is a high
likelihood (in the case of Lattice QCD it is virtually a certainty) that data
is in some way distributed amongst the numerous QCDSP nodes. The question
then is, how to collect data from the nodes of the QCDSP so that it 
can be written to disk, or alternatively, how to get data from a disk
onto the nodes of QCDSP.

A complication is that on the QCDSP only one processor, to which we
shall refer hereafter as the root node (Motherboard 0, Daughterboard 0,
Unique ID=0) is allowed to actually perform I/O onto a disk on the
host computer. The other nodes can output to what in a UNIX system
would be the {\em standard output} stream. This however is merely a
storage buffer on the memory of the node in question, that has to be
retrieved later using the {\bf qprintf} command (as discussed in
section one).  In fact only node 0 can write to the output stream
during run time in a way that it is echoed on the screen of a user or
can be redirected into a file.  The QOS has no concept of other
streams such as the UNIX {\em standard input} and {\em standard
error}. On the other hand the {\bf qrun} command does allow the
passing of {\em command line arguments} to executing programs as will
be demonstrated later.

Hence, one would think that the best way to read a serial file and
distribute its data amongst processors is for the root node to read
the data and perform various scatter operations to distribute the
data.  Alternatively, saving of data could be done by the root
processor gathering data from the other processors and then writing it
out to disk. This is all very well, as long as the root processor can
hold the whole overall dataset in memory. However, a the amount of
DRAM memory on a single node is quite small, a mere
0.5Mwords\footnote{The wordsize of the DSP is 32bits. The physical
size of a word in DRAM is 40bits of which 8 are reserved for error
correction. Hence 0.5Mwords corresponds to 2Mb of useable
memory. However since everything is word size for the DSP from
characters to floating point numbers it makes more sense to discuss
memory in terms of words than bytes.}. Of this the run time operating
system occupies 128Kwords leaving a mere 384Kwords to the user. This
has to include the user code as well as the user data. Hence one
would be reduced to communicating the data from the nodes to the root node, at 
a rate of one pair of processors at a time. Since the QCDSP has no general
point to point communication routines, this process would have to be implemented
using the usual nearest neighbour communications.

Another alternative would be to use a parallel file system. There is 
such a system currently under development for the QCDSP, however it
is not yet an integral part of the QOS, and hence it shall not be 
discussed any further here. 

The question then remains: How can parallel data be loaded and
distributed in a straightforward manner on the QCDSP, and how can the
data be saved from amongst all the processors into one file. On the
QCDSP this is done via the {\bf qread} and {\bf qload} commands using
objects called Node Tagged Tiles (NTFs). An NTF is a single serial
file written in ASCII (can we do binary Bob?) that contains the
relevant data from all the nodes of the QCDSP. Preceeding the data
from a given node is a tag identifying the node which is to host the
data. We will discuss their detailed. One can then develop
workstation tools to convert between these NTFs and regular serial
files. For those readers who intend to load and save lattices to and
from the Columbia Physics environment (stored in the Gauge Connection
Archive Format\footnote{See the Gauge Connection Web site {\tt http://qcd.nersc.gov}}) QOS provides the more specialised commands: {\bf
qload\_lattice} and {\bf qunload\_lattice}. 

\section{Parallel I/O with NTFs, qread and qload}
When a QCDSP program exits and the QOS returns control to the front end,
the memory space of the program prior to exit is left untouched. The 
{\bf qread} and {\bf qunload\_lattice} routines access this memory directly
and dump a user specified number of data blocks from a 
user specified memory address to a file. The {\bf qread} command can 
dump either to a screen or to a NTF. Its output is always in ASCII
format. The {\bf qunload\_gauge} command writes directly to a file
in the Gauge Connection Archiver format.

The {\bf qload} command can be used to place data into processor
memory prior to execution. It works in two ways:
\begin{itemize}
\item
One can specify a memory address and the value to be placed there
as command line arguments to {\bf qload}. The {\bf qload} command will
then place the given value into the given memory address. The user program
can then be run. It can be passed the address of the data and the number
of data blocks as command line parameters. Once running the code can set
a pointer to the given address and copy the relevant data to some 
of its own memory.
\item
The qload command can be given the name of an NTF. The NTF contains
the starting address and the number of blocks for the data,as well as the
data itself, for each processor.
The program can access the data by setting a pointer to the  address.
which can be passed as a command line argument.
\end{itemize}

The {\bf qload\_lattice} command works similarly to the {\bf qload}
command. However the gauge connection format lattice file is not node tagged. 
The user passes the address where the first lattice site is to be placed
on each processor, and the name of the lattice file to the {\bf qload\_lattice}
command. The lattice sites are then placed into processor memory. A 
lattice object can then be instantiated giving the address of the first
lattice element to the lattice object constructor.

Of course in both the above cases dealing with {\bf qload}, the data
may be copied directly into the program data area. This can be made to
be safe, as the linker produces resolved code. This means that the
code produced by the linker will not be relocated afterwards at the
start of execution. Hence one can print out the address of a declared
array in the program, safe in the knowledge that the next time the
program will run, the array will have the same address. {\bf WARNING:
This behaviour is peculiar to the QCDSP.} Furthermore one can glean
information about the addresses of symbols after the compilation and
linking by looking through the linker map file. We shall discuss this
in more detail in the next chapter.

Before describing the details of the NTFs, and giving examples of using
the {\bf qload} and {\bf qread} lattices we should say a few words about
the organisation of memory on each node of the QCDSP. Even before that
here is possibly our most severe {\bf WARNING: The command {\bf qload}
can (and will) if incorrectly used, overwrite memory used by  the QOS 
operating system. This can have UNPREDICTABLE results ranging from no
effect, through subtle bugs, to system crashes. Be careful with it!!!}.

\subsection{The DSP memory Map Part 1}
The complete memory map of the DSP is quite complicated. In this section
we shall just describe a small part of it, that pertains to compiling
and running programs. 

The memory addresses on a single QCDSP node can range from 000000 to 
FFFFFF (remember your hexadecimal numbers?). This space however is 
not all memory and some of it is not accessible. In particular 
the 0.5Mwords of DRAM are mapped between addresses 001000
to 07FFFF. The addresses from 000000 to 001000 are not accessible by 
the DSP in the mode it is used. Memory addresses 080000 and upwards
are special. They either contain views of the data in real memory
(ie that between 000000 and 07FFFF) for use by other hardware such
as the circular buffer, or they contain the memory mapped
images of the DSP internal memory and control registers. We may
deal with some of these latter in the chapter on optimisation.

The run time operating system starts at 060000 and can stretch
until 07FFFF. Consequently {\bf only the memory area from address
001000 to 05FFFF (inclusive) is available for user programs.}

This rough sketch of the memory map is illustrated in figure 
\ref{f:MMap1}.

\begin{figure}[h]
\begin{center}
\leavevmode
\hbox{
\begin{tabular}{|c|l|} 
\hline 
{\bf Address Range} & {\bf Function } \\
\hline
D00000 -- FFFFFF & \begin{tabular}{l} Used on node 0. Unused elsewhere. Generally reading \\ from or writing to here will cause the DSP to hang \end{tabular} \\
\hline 
880000 -- CFFFFF & \begin{tabular}{l} Various images of the 0.5Mwords of DRAM between \\  000000 -- 07FFFF for use of the circular buffer \end{tabular}  \\ 
\hline
820000 -- 87FFFF & Not used \\
\hline 
810000 -- 81FFFF & Addresses of registers for controlling the NGA \\
\hline
800000 -- 80FFFF & Inernal DSP Memory and control register area \\
\hline 
080000 -- 7FFFFF & \begin{tabular}{l} Images of 0.5Mwords of DRAM between 000000 -- 07FFFF \\  for use of the circular buffer \end{tabular} \\
\hline 
{\bf 060000 -- 07FFFF} & {\bf 128Kwords of DRAM reserved for Operating system} \\
\hline
{\bf 001000 -- 05FFFF} & {\bf 384Kwords of DRAM for User programs} \\
\hline
 000000 -- 000FFF & Inaccessible \\ \hline 
\end{tabular}}
\end{center}
\caption{Memory Map of a DSP node. The Memory is word addressed.}
\label{f:MMap1}
\end{figure}

\subsection{Program Images} 
If you have been using the default Makefiles you should find that 
usually a file is produced with the {\bf .map} suffix alongside your
executable. In the case of the ``Hello Wold'' program this file would
be called something like {\tt hello\_world.out}. (The default makefile
always names output files after the directory in which the source files live).

The map file is quite useful as it tells you about how much memory 
your program has allocated statically during compilation, allowing you
to figure out how much space you have to {\em qload} your NTF into. 

Your program, as far as the linker is concerned consists of 6 parts 
of which 3 are negligible. The important three parts go by the names of
{\em data segment}, {\em binary stack segment (BSS)} and the {\em text segment}.
The unimportant parts are there mostly to supply information to debugging tools.
In the case of the Tartan Linker these are called the {\em debug directives} 
segment, the {\em debug strings} segment and the {\em debug source location}
segment. You need not care about these right now. Unless you switch on the
option on the compiler to produce debugging information, these latter three
segments will be empty.

The BSS contains amongst other things your runtime stack, on 
which C++ allocates automatic variables. The data segment is used
to hold various constant data items and the text segment holds the
actual machine level instructions of your program. Any free memory outside
these segments is unallocated and should be placed onto the heap when your
program starts running so that you can allocate memory from it dynamically.

The map file tells you about how your program is laid out in memory.
It starts off by telling you how much space is allocated to the various
segments. For example here are some excerpts from my {\bf hello\_world.map}
file

{\small
\begin{verbatim}
Tartan Linker, SPARC/C40, v5.1.0        4/4/100 13:33:37                Page   1
Copyright (c) 1986-1992 by Tartan, Inc., All Rights Reserved


Allocation to Output Section ".bss" in module *unnamed* in file
"/homeqs0/bj/hello_world/hello_world.outtof"

  Offset   Length               Input Section

    1000      162       "TL.Init" in module "main" in file
                        "/homeqs0/bj/hello_world/main.tof"
    1162        1       ".bss" in module "qcdsp_tcroot" in file
                        "/usr/local/tartan/v2.1/etc/qcdsp_v5.3.3/tcrt30bs.olb"
    1163       21       "OWN" in module "inifin" in file
                        "/usr/local/tartan/v2.1/etc/qcdsp_v5.3.3/tcrt30bs.olb"
    1184     1000       ".stack" in file
                        "/usr/local/tartan/v2.1/etc/qcdsp_v5.3.3/link.lcf"

  Total Allocation = 1184 (hex)

Allocation to Output Section ".data" in module *unnamed* in file
"/homeqs0/bj/hello_world/hello_world.outtof"
  Offset   Length               Input Section

    2184        D       "DEFALT" in module "qcdsp_tcroot" in file
                        "/usr/local/tartan/v2.1/etc/qcdsp_v5.3.3/tcrt30bs.olb"
    2191      16E       ".cinit" in file
                        "/usr/local/tartan/v2.1/etc/qcdsp_v5.3.3/link.lcf"

  Total Allocation = 17B (hex)

Allocation to Output Section ".text" in module *unnamed* in file
"/homeqs0/bj/hello_world/hello_world.outtof"

  Offset   Length               Input Section

    22FF       6C       "_main" in module "main" in file
                        "/homeqs0/bj/hello_world/main.tof"
    236B       2A       ".text" in module "sysfunc" in file
                        "/usr/local/tartan/v2.1/etc/qcdsp_v5.3.3/tcio30bs.olb"
    2395       56       ".text" in module "qcdsp_tcroot" in file
                        "/usr/local/tartan/v2.1/etc/qcdsp_v5.3.3/tcrt30bs.olb"
    23EB       26       "dbgrpc" in module "tcrpc" in file
                        "/usr/local/tartan/v2.1/etc/qcdsp_v5.3.3/tcrt30bs.olb"
    2411        7       ".text" in module "tcrpc" in file
                         "/usr/local/tartan/v2.1/etc/qcdsp_v5.3.3/tcrt30bs.olb"
    2418       32       ".text" in module "tcinit" in file
                        "/usr/local/tartan/v2.1/etc/qcdsp_v5.3.3/tcrt30bs.olb"
    244A       22       "_exit" in module "inifin" in file

Tartan Linker, SPARC/C40, v5.1.0        4/4/100 13:33:37                Page   2
Copyright (c) 1986-1992 by Tartan, Inc., All Rights Reserved

                        "/usr/local/tartan/v2.1/etc/qcdsp_v5.3.3/tcrt30bs.olb"
    246C        2       "__main" in module "inifin" in file
                        "/usr/local/tartan/v2.1/etc/qcdsp_v5.3.3/tcrt30bs.olb"

  Total Allocation = 16F (hex)
\end{verbatim}
}
You can clearly identify the three sections for the BSS, data and text segments.
(They are referred to as output sections {\tt .bss}, {\tt .data} and {\tt .text}.)

Following all this information is a summary of the output sections that
tells you the total amount of memory allocated to each individual section.
In my {\bf hello\_world.map} file this looks like:
{\small
\begin{verbatim}

                Output Sections

Section List for Module *unnamed* in file
"/homeqs0/bj/hello_world/hello_world.outtof"

Number  Physical   Start   Length  Kind         Access  Name
  1                  1000     1184 data                 ".bss"
  2                  2184      17B constant             ".data"
  3                  22FF      16F code                 ".text"
  4                     0        0 debug                "debug_directives"
  5                     0        0 debugstring          "debug_strings"
  6                     0        0 line number          "debug_source_location"
}
\end{verbatim}}
This is perhaps the most useful piece of information in the map file.
It tells me the starting addresses of each of my program segments and 
how long they are (the columns Start and Length). I can see from here that 
my program uses memory from address 001000 to 0022FF + 16F = 00246E inclusive.
Hence if this program were to take any input via {\tt qload} I would 
know that it should be safe to fill up memory from 0246F upwards.

The remainder of the map file lists every single symbol used in your
program. You may find it amusing to see how few of them you recognise.
Here are some symbols from the {\bf hello\_world.map} file that you 
may recognise
{\small
\begin{verbatim}
Kind  Offset Sect Virtual Physical      Name
          6F   3     236E          "_CoorT"
          70   3     236F          "_CoorX"
          71   3     2370          "_CoorY"
          72   3     2371          "_CoorZ"
          93   3     2392          "_CurrentStatus__FiPc"
 W        97   3     2396          "_c_int00"
          6D   3     236C          "_DbNum"
         14C   3     244B          "_exit"
         14B   3     244A          "_exit$LAJ"
          8E   3     238D          "_fclose"
          8C   3     238B          "_fopen"
          8D   3     238C          "_fprintf"
          91   3     2390          "_HdwCheck__Fi"
          94   3     2393          "_InterruptExit__FiPc"
           1   3     2300          "_main"
           0   3     22FF          "_main$LAJ"
          6C   3     236B          "_MbNum"
          90   3     238F          "_NodeStatUpdate__Fv"
          77   3     2376          "_NumNodes"
          7C   3     237B          "_printf"
\end{verbatim}
}
You may be worried here that at startup, spurious memory allocation
might overwrite your qloaded data. This is not a worry here as the
map file includes the heap area as well. Hence if you place something
above the last address in the map file, it is guaranteed that it 
will not be malloced over. 

You can configure the size of the heap and other program layout
options in a file called the Linker Control File. If you do not 
use one of these the program will be linked under the control of 
a default linker control file which allocates a pitifully small heap.
(0x1000 words). Hence when you come to code bigger projects you
will have to allocate the heap yourself in your own LCF. This will
involve a partitioning of the available memory between room for 
your program, data you intend to allocate dynamically (heap) and data
you intend to qload. 

The standard procedure is to qload data very high in memory, just
below the operating system. Suppose you have data of say 32Kwords per
processor (0x8000 words (hex)). The operating system starts at address
0x060000 and so you might want to load the data to 0x06000 - 0x008000
= 0x058000. Then you can set up your LCF so that your program cannot
extend beyond 0x057fff (heap included). It is of course prudent to
leave a bit of room here and there in case you modify your program. In
the words of investment brokers ``The size of your program can go down
as well as up.''.

We will discuss linking, program sections and control files in more detail
in the next chapter about compiling and linking.

\subsection{Node Tagged File Formats}
We now discuss node tagged files.
Node tagged files are written in ASCII and contain hexadecimal numbers.
Although this can cause the files to be large and slow to read and write,
it does have the advantage that the files are readable without needing
specialist dump programs.

The file format consists of one hexadecimal entry on each line 
written in ASCII. The structure of the entries is given below.
below. Each heading and sub--heading corresponds to an entry 
in the NFT.
\begin{description}
\item{\bf A 'Magic Number' : \ } -- This is an integer for identifying
which of the 3 types of node tagged files one is dealing with. The
three types are: \begin{description} \item{tree \ } -- the nodes are
tagged according to their position on the SCSI tree by a pair of
numbers (M,D) representing the node's motherboard and daughterboard
number respectively. Each of these numbers corresponds to a new entry
on a new line in the NTF. The magic number for this mode is 2.  This
is the 'default' tag mode. Node tagged files in this format usually
have a {\bf .tree} file suffix.  \item{Machine 4D \ } -- the nodes are
tagged with their 4D coordinates within the processor grid (A
four-tuple of integers, each of these on a new line in the NTF). The
magic number for this tag mode is 3. Node tagged files in this mode
usually have a {\bf .m4d} file suffix.  \item{Physics 4D \ } -- to all
intents and purposes this format the same as the machine 4D. Nodes are
tagged by their 4D coordinates. The meaning of fields however is
different. In terms of machine coordinates, everything is
fixed. However the machine coordinates can be re mapped at boot time to
interchange the order of coordinates. Hence (T,X,Y,Z) coordinates can
be re mapped to (X,Y,Z,T) coordinates etc. The machine 4D file never
changes the data layout. However the physics 4D file can shift the
data around depending on the current 4D mapping. The magic number for
this mode is 4. Node tagged files in this mode usually have the {\bf
.p4d} file suffix.  \end{description}
\item{\bf A sequence of Node Records} -- for each node in the processor there is
a record. A record has the following structure.
	\begin{description}
	\item{A Node Tag \ } -- A tag identifying the node. For the tree structure this consists of the  motherboard number followed by the daughterboard
number. For the 4D modes this is a four tuple of integers identifying the
4D coordinates of the node.
	\item{The number of blocks for the Node \ } -- An integer indicating
how many blocks of data are to be loaded into the memory of the node (Hexadecimal without leading 0x)
	\item{Base Address \ }--  A hexadecimal address (without leading 0x) specifying the location in memory where the first data item is to be placed.
	\item{Data Blocks \ } -- An ASCII representation of each data block. In the case of integers this is quite readable. In the case of floats it is not
	\item{A terminator \ } -- An ASCII zero follows the final record as a 
record termination signal.
	\end{description}
\end{description}

The formats of the NTFs, the magic numbers, node tags and file extensions
are summarised in figures \ref{f:NTFStructure}, \ref{f:NTFRecStructure} 
and \ref{f:NTFTypeSummary}.

\begin{figure}[ht]
\begin{center}
\leavevmode
\hbox{%
\begin{tabular}{|c|}
\hline
Magic Number \\
\hline
Record No 1 \\
\hline
Record No 2 \\
\hline 
$ \vdots $ \\
\hline
Last Record \\
\hline
\end{tabular}}
\end{center}
\caption{Structure of a Node Tagged file -- Components are sequential in an ASCII file. Each component starts on a new line}
\label{f:NTFStructure}
\end{figure}

\begin{figure}[ht]
\begin{center}
\leavevmode
\hbox{%
\begin{tabular}{|c|} \hline
Node Tag \\
\hline 
$n$ -- Number of blocks \\
\hline 
Base Address \\
\hline 
block[0] \\
\hline
block[1] \\
\hline
$\vdots $ \\
\hline
block[ $n-1$ ] \\
\hline
0 (terminator) \\
\hline
\end{tabular}}
\end{center}
\caption{Structure of a Node tag record -- Each component is in ASCII on a new line}
\label{f:NTFRecStructure}
\end{figure}

\begin{figure}[ht]
\begin{center}
\leavevmode
\hbox{%
\begin{tabular}{|c|c|c|c|}
\hline 
{ \bf File Type}  & {\bf Magic Number} & {\bf Node Tag Components} & {\bf File Suffix } \\
\hline
Tree & 2 & \begin{tabular}{c} Motherboard No \\ Daughterboard No \end{tabular} & {\tt .tree} \\
\hline
Machine 4D & 3 & \begin{tabular}{c} T Coordinate \\ X Coordinate \\ Y Coordinate \\ Z Coordinate \end{tabular} & {\tt .m4d} \\
\hline
Physics 4D & 4 &  \begin{tabular}{c} T Coordinate \\ X Coordinate \\ Y Coordinate \\ Z Coordinate \end{tabular} & {\tt .p4d} \\
\hline
\end{tabular}}
\end{center}
\caption{Magic Number Node Tag and File Suffix information for the three node tagged file types. The magic number is the first number of any node tagged
file. It should be an ASCII integer on the first line. The node tag
fields should be ASCII integers. Each entry should take a separate
line in the file.}
\label{f:NTFTypeSummary}
\end{figure}

Perhaps the easiest way of creating a node tagged file is to dump
some memory from a program. We now proceed to demonstrate how this is done.

\section{Creating Node Tagged files using qread}
Consider the following little program:
{\scriptsize
\begin{verbatim}
#include <stdio.h>
#include <stdlib.h>
#include <strings.h>
#include <sysfunc.h>


int main( int argc, char *argv[] ) 
{
  float i[5]={1.0,2.0,3.0,4.0,5.0};

  printf("The address of i is %x and its length is %x blocks\n",
           i, 5*sizeof(int));

  return(EXIT_SUCCESS); 
}
\end{verbatim}
}

The code itself does nothing, but prints out the address of the first 
element of the floating point number array {\tt i}. We can use this 
information to dump the floating point array into an NTF.

\subsection{Setting options for qread}
There are several options that allow you to control how you want to
perform the dump. For example, you can dump to a screen or a named 
file. You can dump into an NTF of either the tree, machine 4D or 
of the physics 4D variety. You can rewrite an existing file or 
append to it. These options are set by the following {\tt qcsh} commands:
\begin{description}
\item{{\tt qset\_read\_output\_filename} : \ } This command takes one 
argument which is the name of the file you want {\tt qread} to 
dump your data to.
\item{{\tt qset\_read\_output\_file\_access} : \ } This command 
takes a single argument to determine whether you wish to use
the append write mode or the overwrite mode. The possible arguments
are 
	\begin{description}
	\item{\tt a \ } -- append mode
	\item{\tt w \ } -- write mode
	\end{description}
\item{{\tt qset\_read\_output\_select } : \ } This command determines 
whether you wish your dump to go to the screen or to a named file
(that you have to name using {\tt qset\_read\_output\_filename}.
It takes a single argument. The argument can have the following values:
\begin{description}
	\item{\tt f \ } -- dump to named file
	\item{\tt s \ } -- dump to screen
\end{description}
\item{{\tt qset\_read\_output\_tagged} : \ } This command determines
the tag type of your dump file. It takes a single argument. Allowed 
argument values are
\begin{description}
\item{ \tt no \ } -- Do not node tag the file. I do not describe the
results of this option. 
\item{ \tt yes \ } -- Place node tags into file. Use default tag mode (tree).
\item{ \tt tree \ } -- Use tree tag mode.
\item{ \tt m4d \ } -- Use machine 4D tag mode.
\item{ \tt p4d \ } -- Use physics 4D tag mode.
\end{description}
\end{description}

\subsection{Using qread to dump to a file}
Let us now have a go at dumping the floating point array in the previous
code snippet. Create a directory called {\tt dump} somewhere and copy
the program listing above into a file named {\tt main.C} (alternatively
the directory may be available under {\tt <QOS\_VERS>/examples/ParIO/dump}).

If you are doing things by hand, you probably want to copy a 
default {\tt Makefile} from the {\tt hello\_world} directory and
build the program by typing {\tt make}. This should produce a program
called {\tt dump.out}. If you are copying the directory structure it should
have a Makefile with it.

Boot the machine and enter the QC-Shell.I will use the machine {\tt
q\_1}.  Now let us suppose that we want to dump the {\tt float} array
from each PE into a file called {\tt foo}.  Let us try first with the
tree node tag mode. We set up the dump using the following {\tt qcsh}
commands (I removed the Qdaemon responses to save space).
\begin{verbatim}
(dump: qcsh[q_1])% qset_read_output_select f
(dump: qcsh[q_1])% qset_read_output_filename foo.tree
(dump: qcsh[q_1])% qset_read_output_file_access w
(dump: qcsh[q_1])% qset_read_output_file_tagged tree
\end{verbatim}

Now run the program using {\tt qrun} as before (Qdaemon messages removed):
\begin{verbatim}
(dump: qcsh[q_1])% qrun dump.out
The address of i is 105d and its length is 5 blocks
\end{verbatim}
Now I can dump to the NTF using {\tt qread}. This command takes the 
base memory address and the number of blocks as two hexadecimal numbers
respectively (without the leading 0x in terms of notation). I dump the 
array with the command (of course you may find a different base 
address to 105d in which case you have to substitute that in what
follows) :
\begin{verbatim}
(dump: qcsh[q_1])% qread 105d 5
\end{verbatim}
to which the machine responds:
\begin{verbatim}
Qdaemon state is:
        qdaemon task is READ
        qdaemon abort no and resume no
        QCDSP synchronous
        All nodes selected with SCSI tree coordinates

Qdaemon:  Parameters for read from QCDSP:
        Read address:           0x105d
        Read blocksize:         0x5
        Filename:               foo.tree
        File access:            write
        File format:            node tagged tree


Qdaemon state is:
        qdaemon task is READ
        qdaemon abort no and resume no
        QCDSP synchronous
        All nodes selected with SCSI tree coordinates
\end{verbatim}
and the file {\tt foo.tree} magically appears in the directory.

We can actually look at the {\bf foo.tree} file. My one looks like:
{\small
\begin{verbatim}
2
0
0
5
105d
0
1000000
1400000
2000000
2200000
0
0
1
5
105d
0
1000000
1400000
2000000
2200000
0
.
.
.
0
3f
5
105d
0
1000000
1400000
2000000
2200000
0
\end{verbatim}
}

You can clearly see the structure here. The first number in the file
is the magic number 2. The next two are the node tag for the tree
coordinates.  These are motherboard and daughterboard identifiers (the
last daughterboard identifier is 3f which is hex for 63.  This is
because I am using only a 64 PE single motherboard machine).  After
this you have the number of blocks -- 5 -- followed by the base
address 105d. You have 5 lines of data. They are the hex
representations of the bit pattern that makes up our 5 floating point
numbers. Finally you have a 0 as an end of record marker.

\section{Loading data with qload}
\subsection{Poking individual words with qload}
The command {\tt qload} allows you to either load node tagged files or to place
individual words into memory. Like qread it also has a number of options
you can set. In this subsection we discuss first how to place individual 
words in memory (ever hear of the POKE keyword in BASIC? This is kind of
the same stuff).

The incantation to load a word into a particular location in memory is
of the format 
\begin{verbatim}
qload <address> <data>
\end{verbatim}
where both the address and the data are hexadecimal values. For example
to load the value 5 into the memories of all the processors at address
04EFFF I would type a magic incantation like:
\begin{verbatim}
(qcdhost/homeqs0/bj/QCDSP/IO/load: qcsh[q_1])% qload 04EFFF 5
\end{verbatim}

To that this has worked consider the following program:
{\scriptsize
\begin{verbatim}
#include <stdio.h>
#include <stdlib.h>
#include <strings.h>
#include <sysfunc.h>

int main( int argc, char *argv[] )
{
  // -----------------------------------------------
  // qload will put (poke?) the distributed memory
  // into the data. User passes in the address the
  // data has been load to through command line
  // arguments
  // -----------------------------------------------
  int *loaded_ints;
  int nblocks;

  // -----------------------------------------------
  // Check command line arguments
  // -----------------------------------------------
  if( argc != 3) {
    printf("Usage: qrun load.out <base address(in hex)> <no of blocks(in hex)>\n
");
    return(EXIT_FAILURE);
  }

  // -------------------------------------------------------------
  // Convert HEX base address string into an integer pointer value
  // and point loaded_ints there
  // -------------------------------------------------------------
  loaded_ints = (int *)strtol(argv[1],  (char **)NULL, 16);

  // -------------------------------------------------------------
  // Convert no_of_blocks string into an integer value
  // -------------------------------------------------------------
  nblocks = (int)strtol(argv[2], (char **)NULL, 16);

  // -------------------------------------------------------------
  // Confirm values to the user
  // -------------------------------------------------------------
  printf("%x (%d) blocks of data at %x\n", nblocks, nblocks,
         (unsigned int)loaded_ints);

  // -------------------------------------------------------------
  // Print back the data
  // -------------------------------------------------------------
  for(int i = 0; i < nblocks; i++) {
    printf("%d ",loaded_ints[i]);
  }
  printf("\n");

  return(EXIT_SUCCESS);
}
\end{verbatim}
}

This program basically checks your {\tt qload}.You need to  pass it the base
address to where you have loaded your data as well as the number of blocks 
comprising the data. These are passed on to the program as strings
in the {\tt argv} array, so first we have to convert them to addresses.
This is done using the {\tt strtol} function call which converts a 
string to a long integer.

The number 16 in the third argument tells {\tt strtol}\footnote{{\tt
strtol} is a very useful function. For details look at the {\tt man}
page in section 3 of the manual} that the string is supposed to
represent a hexadecimal number. We set a pointer {\tt loaded\_ints} to
the results of the string conversion, in other words to our base
address.

We then also need to convert the number of blocks to an integer. This is
done by the second call to {\tt strtol}. 

After confirming to the user that the input values have been read correctly,
the code proceeds to list the specified number of words from the base 
address, treating them as integers and prints them to the screen.

Compile up this little program. You should also find it already written
on the QCDSP in the directory {\tt <QOS\_VERS>/examples/ParIO/load} with
a suitable default Makefile (Replace this with wherever we put it...)
Now lets have a look in the map file (in my case its called {\tt load.map}):
{\scriptsize
\begin{verbatim}
                Output Sections

Section List for Module *unnamed* in file
"/homeqs0/bj/QCDSP/intro/IO/load/load.outtof"

Number  Physical   Start   Length  Kind         Access  Name
  1                  1000     110C data                 ".bss"
  2                  210C      102 constant             ".data"
  3                  220E      26F code                 ".text"
  4                     0        0 debug                "debug_directives"
  5                     0        0 debugstring          "debug_strings"
  6                     0        0 line number          "debug_source_location"
\end{verbatim}
}
This tells me that it is safe for me to load things above the address 220E+26F = 247D. However it is good practice to load things high in the memory so I will
load the data to 05FFE0. I do this by typing
\begin{verbatim}
(qcdhost/homeqs0/bj/QCDSP/intro/IO/load: qcsh[q_1])% qload 05FFE0 5
\end{verbatim}
Now I run my program 
\begin{verbatim}
(qcdhost/homeqs0/bj/QCDSP/intro/IO/load: qcsh[q_1])% qrun load.out 05FFE0 1
\end{verbatim}
The computer replies
\begin{verbatim}
1 (1) blocks of data at 5ffe0
5 
\end{verbatim}
on all the processors.

\subsection{Loading NTFs using qload}
The qload command can also be used to load NTFs. Since all the information
about the data is in the NFT (processor coordinates, base addresses, number of blocks and the actual data) is present in the NTF file. There is no need
to specify these. The command is simply
\begin{verbatim}
qload <filename>
\end{verbatim}
The command can identify the tag type used in the file from the magic number.

Consider the little program above but now converted to dump floats instead 
of ints:
{\scriptsize
\begin{verbatim}
#include <stdio.h>
#include <stdlib.h>
#include <strings.h>
#include <sysfunc.h>

int main( int argc, char *argv[] )
{
  // -----------------------------------------------
  // qload will put (poke?) the distributed memory
  // into the data. User passes in the address the
  // data has been load to through command line
  // arguments
  // -----------------------------------------------
  float *loaded_floats;
  int nblocks;

  // -----------------------------------------------
  // Check command line arguments
  // -----------------------------------------------
  if( argc != 3) {
    printf("Usage: qrun load.out <base address(in hex)> <no of blocks(in hex)>\n
");
    return(EXIT_FAILURE);
  }

  // -------------------------------------------------------------
  // Convert HEX base address string into an integer pointer value
  // and point loaded_ints there
  // -------------------------------------------------------------
  loaded_float = (float *)strtol(argv[1],  (char **)NULL, 16);

  // -------------------------------------------------------------
  // Convert no_of_blocks string into an integer value
  // -------------------------------------------------------------
  nblocks = (int)strtol(argv[2], (char **)NULL, 16);

  // -------------------------------------------------------------
  // Confirm values to the user
  // -------------------------------------------------------------
  printf("%x (%d) blocks of data at %x\n", nblocks, nblocks,
         (unsigned int)loaded_ints);

  // -------------------------------------------------------------
  // Print back the data
  // -------------------------------------------------------------
  for(int i = 0; i < nblocks; i++) {
    printf("%f ",loaded_floats[i]);
  }
  printf("\n");

  return(EXIT_SUCCESS);
}
\end{verbatim}
}

We can use this to load our floating point NFT ({\tt foo.tree}) that 
we created before, but first another word about memory.

\subsection{Manual Data Relocation}
Usually it will be the case, that a program loads an NTF at the start
to a memory location, then processes it and then dumps it at the end.
In this case the address in the node tag file is fine both for reading 
and dumping. However we are now at an advanced stage, where one program
produced the NTF but another program wants to load it.

 In the program that made the NFT, the data was in a standard
statically defined array that started (in my case) at address
105d. Now however, we want to load it into a different program.  After
converting the dump program to deal with floats, a quick look at the
map file tells me that memory is still free from above 247D -- this is
a consequence of the fact that the {\tt int} on the DSP is the same
size as the {\tt float}. It is desirable to load the data to a high address,
say 05e000. This can be achieved by editing the NTF and replacing 105d
everywhere with 05e000. In my case the it is simply a question of a global
edit. This is one of the advantages of the NTF being in ASCII format.

\subsection{Example of loading an NTF}
So, edit the previous program to deal with floats as I have done above.
Edit your file {\tt tree.foo} replacing the base addresses everywhere with
05e0000. Load the NTF using the command
\begin{verbatim} 
(qcdhost/homeqs0/bj/QCDSP/intro/IO/load: qcsh[q_1])% qload foo.tree
\end{verbatim}
and run the program, it should probably be called {\tt load.out}:
\begin{verbatim}
(qcdhost/homeqs0/bj/QCDSP/intro/IO/load: qcsh[q_1])% qrun load.out 05e000 5
\end{verbatim}
The computer should reply:
\begin{verbatim}
5 (5) blocks of data at 5e000
1.000000 2.000000 3.000000 4.000000 5.000000 
\end{verbatim}
on all the processors.

\subsection{Setting up qload options}
You should now be able to use {\tt qload} to load parallel data. However
just like the case of {\tt qread} there are some options that you can
set. These allow you to specify the filename to {\tt qload} or simply 
just the address and value to {\tt qload} (for poking single data). Then 
you can use {\tt qload} with no arguments at all. The commands to set 
the options are
\begin{description}
\item{ {\tt qset\_load\_select} : \ } Specifies whether {\tt qload} should
read from a file or from the screen. The command can have up to 1 argument
whose value is either {\tt file} or {\tt screen} (alternatively {\tt f} or
{\tt s} respectively). If no arguments are given, the current value of
the option is shown. I generally found that this option reset itself to 
file every time {\tt qload} is invoked with an explicit filename.
\item{ {\tt qset\_load\_screen} : \ } Specifies an address and a data 
item. This option sets things up so that a subsequent call to {\tt qload} 
will load the specified data to the specified address. The command can 
have either  two arguments or none. Invoking the command with no arguments
causes it to print the current values for the address and the data value.
When the command is invoked with two arguments, the meaning of the arguments
are
\begin{description} 
\item{\tt address \ } specifies the address to load to.
\item{\tt data value \ } specifies the value of the data to be loaded.
\end{description}
Notes:
\begin{itemize}
\item 
The following are equivalent: {\tt qload <address> <value>} and {\tt qset\_load\_select s ; qset\_load\_screen <address> <value> ; qload}.
\item
If one selects to load from the screen, and sets the address and value and
then invokes {\tt qload <filename>}, then {\tt qload} will switch to 
reading from the file instead. If one has set a filename to load from 
using {\tt qset\_load\_filename} and has switched {\tt qload} to read from 
file using {\tt qset\_load\_filename file}, if one now executes {\tt qload <address> <value>} then {\tt qload} will switch to screen mode and the last option
will be valid.
\end{itemize}
\item{\tt qset\_load\_filename : } This command sets the name of the file
that qload will use to load data from. It can have either one or no arguments.
If the command is invoked with no arguments it will echo the current value
of the filename. If invoked with one argument the argument, the argument
should be the name of the file to load from.
Note: The following are equivalent: {\tt qload <file>} and {\tt qset\_load\_select file ; qset\_load\_filename <file> ; qload }.
\end{description}

\section{Digression for Lattice Folk}
There are two specialised commands to load and unload lattice gauge fields to
and from the QCDSP. These are, as mentioned previously {\tt qload\_lattice}
and {\tt qunload\_lattice}. Before describing these I should say a few 
words about the lattice file format.

\subsection{Gauge Connection Format}
Currently NERSC provides an archive area called the {\tt Gauge Connection}.
They use a particular format for lattices there. This format consists
of a header section and a data section. The header section consists of
a whole lot of fields. This is useful for keeping track of where the 
lattices came from. The header file also keeps information about the
average plaquette to assist in validation.

The command {\tt qload\_lattice} simply takes a configuration in the NERSC
archive and loads it into memory at a user supplied starting address.
For example, prior to running a certain piece of code. I have obtained
(actually output from another program) a lattice called {\tt lattice.ieee.full }. The {\tt .ieee} part of the naming convention indicates that the floating point format of the lattice is IEEE 32Bit big--endian. The {\tt .full} part indicates that all the link matrices are stored in full (as opposed to compressed 2 row
format).

I would like to load this lattice prior to computation. It has a volume
of $8^4$ sites, which is to be distributed over 64 processors. My processor
grid is a $4 \times 2 \times 2 \times 4$ lattice (the order of the dimensions
in this case is $X \times Y \times Z \times T$. This implies that the local 
lattice size per processor is $2 \times 4 \times 4 \times 2 = 64$ sites. On
each site there are 4 link matrices each of these being a full SU(3) matrix
consisting of 9 complex numbers which are represented as 18 real numbers.
Hence the size of the local data is $64 \times 4 \times 18 = 4608$ numbers
and since each number is word sized this means that the data requires 
4608 words of storage which in hexadecimal is 1200. Thus the highest I can
load this data in memory is to address 060000 - 001200 = 5ee00. To be 
a little conservative I choose to load it to 5ed00.

The instruction I give to the QCDSP is:
\begin{verbatim}
(qcdhost/homeqs0/bj/QCDSP/sfw/hello_world: qcsh[q_1])% qload_lattice 5ed00 lattice.ieee.full
\end{verbatim}
Apart from the usual qdaemon messges, the response of the computer was 
as follows:
{\scriptsize
\begin{verbatim}
DoLoadLattice:  a copy of the header from lattice.ieee.full
        BEGIN_HEADER
        CHECKSUM =  1649a43
        LINK_TRACE = +0.0020214
        PLAQUETTE = +0.5132502
        DATATYPE = 4D_SU3_GAUGE_3x3
        HDR_VERSION = 1.0
        STORAGE_FORMAT = 1.0
        DIMENSION_1 = 8
        DIMENSION_2 = 8
        DIMENSION_3 = 8
        DIMENSION_4 = 8
        BOUNDARY_1 = PERIODIC
        BOUNDARY_2 = PERIODIC
        BOUNDARY_3 = PERIODIC
        BOUNDARY_4 = ANTIPERIODIC
        ENSEMBLE_ID = BALINT_LAT_1
        ENSEMBLE_LABEL = Balint's first test lattice 8x8x8x8 pure gauge beta=6.0
        SEQUENCE_NUMBER = 100
        CREATOR = Columbia
        CREATOR_HARDWARE = QCDSP 
        CREATION_DATE = Fri Apr 28 15:50:26 2000
        ARCHIVE_DATE = Fri Apr 28 15:50:26 2000
        FLOATING_POINT = IEEE32BIG
        END_HEADER

DoLoadLattice:  keywords used to load lattice
        CHECKSUM = 1649a43
        LINK_TRACE = 0.0020214
        PLAQUETTE = 0.5132502
        DATATYPE = 4D_SU3_GAUGE_3x3
        DIMENSION_1 = 8
        DIMENSION_2 = 8
        DIMENSION_3 = 8
        DIMENSION_4 = 8
        FLOATING_POINT = IEEE32BIG

DoLoadLattice:
        Loading 4608 words per node to address 0x5ed00
        Total transfer of 294912 words to 64 nodes of QCDSP
        Starting to read 294912 words from host disk
        Finished loading 1/4 of lattice to QCDSP
        Finished loading 2/4 of lattice to QCDSP
        Finished loading 3/4 of lattice to QCDSP
        Finished loading 4/4 of lattice to QCDSP
        Converting from IEEE32BIG to TIDSP32 on QCDSP
DoLoadLattice:
        Loading plaq.qin to calculate plaq and trace on QCDSP

Output from plaq.qin running on QCDSP
        Machine size (X,Y,Z,T) (4,2,2,4)
        Lattice size per node (2,4,4,2)
        Total lattice size (8,8,8,8)
        Lattice address 0x5ed00

        plaq            0.5132502
        link trace      0.0020214

MkerExecAck:  starting to check program exit status
AnalyzeQCDSP::SCUCheck:  SCUDebug mode 0 doesn't allow analysis
MkerExecAck:  SCUCheck could not be run

DoLoadLattice:
        Plaquette from header and QCDSP (0.5132502) agree to 0.000001
        Link trace from header and QCDSP (0.0020214) agree to 0.000001
        Checksum from header and QCDSP agree (0x1649a43)
\end{verbatim}}
The messages are quite self explanatory. First qload lattice echoed back
the contents of the header file. Then it loaded the lattice, partitioned
it amongst the processors. Thereafter it converted from IEEE 32Bit big
endian format to the DSPs internal format denoted TIDSP32. A separate
program was then run to calculate the plaquette and the average link trace
to validate the data. Finally QCDSP responded indicating that the
lattice has been loaded, that the plaquette and link trace have been
found to be correct to 4dp and that the checksum test has been passed.

At this point one can start one's computation. To be able to use the
lattice in the Columbia Physics System, one has to set the members of a global
instance of a {\tt GlobalJobParameter} class (it is required to have
one of these called GJP declared globally).  One has to set the
private {\tt start\_conf\_kind} member of the {\tt
GlobalJobParameter} class to have value {\tt START\_CONF\_LOAD}, and
one has to set the private {\tt start\_conf\_load\_addr} pointer to
point in memory to first word of the loaded lattice.

Currently this can only be done through the public {\tt Initialize} member 
function of the {\tt GlobalJobParameter} class. This function takes as 
an argument a reference to a {\tt DoArg} class. The {\tt DoArg} class
has public member variables {\tt start\_conf\_kind} and {\tt start\_conf\_load\_addr}. The former of these has to be set to {\tt START\_CONF\_LOAD} 
and the latter to the address of the first word of the loaded data.

The address can be passed to the program as a command line argument, 
making it available to each processor, or it could be read by node 0
using the usual C style file I/O and then broadcast to the rest of the
system. An example snippet of code to set up the lattice may look something
like as follows:
{\scriptsize
\begin{verbatim}
// ------------------------ 
// Standard headers
// ------------------------

#include <stdio.h>
#include <stdlib.h>     // Exit

// ------------------------
// QCDSP Non Physics Includes
// ------------------------

#include <sysfunc.h>

// ------------------------
// QCDSP Physics Includes
// ------------------------

#include <util/include/lattice.h>   //  Lattice Classes
#include <util/include/gjp.h>       //  Global Job Parameters  Class
#include <util/include/verbose.h>   //  Verbose Output 
#include <util/include/error.h>     //  Error Handler Class
#include <alg/include/do_arg.h>     //  Some kind of argument handling class

// ------------------------
// These are always defined Globally
// -------------------------
GlobalJobParameter GJP;
Verbose            VRB;
Error              ERR;

// -----------------------------------
// PE 0 is boss
// -----------------------------------
#define BOSS_ID  0

main(int argc, char *argv[]) 
{
  // ----------------------------
  // Check Arguments 
  // Can call with either no arguments (ordered start) => argc = 1
  // Or with  one argument, being hex address of gauge field 
  // ----------------------------
  if ( argc > 2  ) {
    printf("Usage: qrun prog.out [hex address (no leading 0x)]\n");
    exit(EXIT_FAILURE);
  }

  // ----------------------------
  // Initialize the Global Job Params
  // -----------------------------
  DoArg do_arg;

  int my_id = UniqueID();


  // --------------------------------
  // Global lattice      X  Y  Z  T 
  // --------------------------------
  int g_latt_size[4] = { 8, 8, 8, 8 };
  // --------------------------------
  //  Local lattice      X  Y  Z  T 
  // ---------------------------------

  int l_latt_size[4] = { 2, 4, 4, 2 };

  printf("Local Lattice Size is: %d %d %d %d\n", 
	 l_latt_size[0],l_latt_size[1],l_latt_size[2], l_latt_size[3]);
  

  // -----------------
  // Use Whole Machine
  // -----------------
  do_arg.x_nodes = SizeX();
  do_arg.y_nodes = SizeY();
  do_arg.z_nodes = SizeZ();
  do_arg.t_nodes = SizeT();

  // ----------------------------
  // We are in 4D by I set the 5th dimension anyway
  // ----------------------------
  do_arg.s_nodes = 1;

  // -----------------
  // Set the lattice volume -- Sites per node
  // -----------------
  do_arg.x_node_sites = l_latt_size[0];
  do_arg.y_node_sites = l_latt_size[1];
  do_arg.z_node_sites = l_latt_size[2];
  do_arg.t_node_sites = l_latt_size[3];
  do_arg.s_node_sites = 1;

  // -----------------
  // Set Boundary Conditions
  // -----------------
  do_arg.x_bc = BND_CND_PRD;  // Periodic
  do_arg.y_bc = BND_CND_PRD;  // Periodic
  do_arg.z_bc = BND_CND_PRD;  // Periodic
  do_arg.t_bc = BND_CND_APRD; // Antiperiodic

  // ----------------
  // Set lattice start depending on no of arguments
  // ----------------
  if( argc  == 1 ) {         // No user supplied arguments
     do_arg.start_conf_kind = START_CONF_ORD;
  }
  else {  // User supplied load address
     
     do_arg.start_conf_kind = START_CONF_LOAD;

     // -------------------------------------
     // Hex String to address convertion
     // -------------------------------------
     do_arg.start_conf_load_addr = (Matrix *)strtol(argv[1],(char **)NULL, 16);
  }
  do_arg.start_seed_kind = START_SEED_FIXED;
  do_arg.verbose_level = VERBOSE_RESULT_LEVEL + 1;

  // ---------------------------------------
  // Quenched QCD -- 3 colours, beta = 6
  // --------------------------------------- 
  do_arg.colors = 3;
  do_arg.beta = 6.0;

  // ------------------------
  // Initialise GJP Structure
  // ------------------------

  GJP.Initialize(do_arg);
  
  // -----------------------
  // Set verbosity
  // -----------------------

  VRB.Level(VERBOSE_RESULT_LEVEL+1);

  // ---------------------------
  // Grab a Wilson Gauge Lattice. with possibly wilson fermions in future
  // This gets initialised from GJP
  // ---------------------------

  GwilsonFwilson lat;

  // --------------------------
  // Calculate The Global Trace of the plaquette
  // --------------------------

  Float t;
  Float normal; // Plaquette normalization factor

  normal = g_latt_size[0]*g_latt_size[1]*g_latt_size[2]*g_latt_size[3];
  normal *= 6 * 3; // No of planes * 3 to normalize unit gauge to 1
  t=lat.SumReTrPlaq()/normal;

  // Print Plaquette
  if( my_id == BOSS_ID ) {
    printf("Boss: Startup Gauge Field    --  Sum Tr Plaq = %f\n",(float)t);
  }

  // ---------------------------------------------
  // Do your world beating calculation below
  // ---------------------------------------------


  // ---------------------------------------------
  // Your world beating calculation ends
  // You want to store gauge field perhaps
  // --------------------------------------------- 
  // --------------------------------------------- 
  // Print Gauge Field -- for saving
  // ---------------------------------------------
  printf("Local Gauge field starts at %x\n",(int)lat.GaugeField());
  exit(EXIT_SUCCESS);
}
\end{verbatim}
}

Running this program with no arguments produced the following output
\begin{verbatim}
ocal Lattice Size is: 2 4 4 2
Boss: Startup Gauge Field    --  Sum Tr Plaq = 1.000000
Local Gauge field starts at 137c8
\end{verbatim}
which is the correct output a unit gauge.
I could recover the gauge field by using {\tt qunload\_lattice}
as I have printed out the address of the start of the gauge field data.

Now I load the gauge field as before (I have decided to omit the output
from {\tt qload\_lattice} as I have already included it in full earlier.
I then run the program giving the starting address (5ED00) as my 
first argument. I get the following reply from the QCDSP:
\begin{verbatim}
Local Lattice Size is: 2 4 4 2
Boss: Startup Gauge Field    --  Sum Tr Plaq = 0.513250
Local Gauge field starts at 5ed00
\end{verbatim}
Indicating that I managed to initialize the lattice correctly.

\subsection{Unloading Gauges}\label{s:GaugeUnload}
Now we can try to unload the gauge and keep it. A few things need to be 
borne in mind. It is a local convention, that gauge fields in use here
at Columbia be kept in the DSP native representation ({\tt TIDSP32}) whereas
fields that are to be shipped off to the gauge connection should really be
in the IEEE big--endian format ({\tt IEEE32BIG}).

Also, gauge fields can be kept in compressed format (2 of the 3 rows stored
 only) for each SU(3) matrix or in full format.

While the {\tt qload\_lattice} command can learn all this from the header 
part of the gauge field configuration file, and the user program 
has knows about the lattice through the {\tt GlobalJobParameter} structure,
the {\tt qunload\_lattice} command, running on the front end, knows nothing
about our lattice and has to be told via a parameter file.

The parameter file essentially provides the header information part of the
gauge configuration data file. It consists of an ordered list of entries, 
which we shall describe below. Each entry lives on a single line of the file.
Lines containing no text are ignored. Lines starting with a {\tt \#} character
are treated as comments and ignored. The {\tt \#} character does not 
need to be in the first column of the line. However it may not follow
an entry. There are therefore no trailing comments. There should be 
no trailing spaces following entries. {\bf The order of entries is important.}

The entries of the file, in order are
\begin{description}
\item{\bf T Size \ } -- this is a single integer giving the global lattice size in the T (Euclidean Time direction)
\item{\bf X Size \ } -- this is a single integer giving the global lattice size in the X direction
\item{\bf Y Size \ } -- this is a single integer giving the global lattice size in the Y direction
\item{\bf Z Size \ } -- this is a single integer giving the global lattice size in the Z direction
\item{\bf Boundary condition for T \ } -- This entry specifies the boundary condition for the T direction. The boundary conditions can be either periodic or
antiperiodic. This entry can take the values {\tt PERIODIC} or {\tt ANTIPERIODIC} respectively.
\item{\bf Boundary condition for X \ } -- This entry specifies the boundary condition in the X direction.  The entry can take the values {\tt PERIODIC} or {\tt ANTIPERIODIC}.
\item{\bf Boundary contidion for Y \ } -- This entry specifies the boundary condition in the Y direction.  The entry can take the values {\tt PERIODIC} or {\tt ANTIPERIODIC}.
\item{\bf Boundary contidion for Z \ }  -- This entry specifies the boundary condition in the Z direction.  The entry can take the values {\tt PERIODIC} or {\tt ANTIPERIODIC}
\item{\bf Lattice Base Address \ } -- The address in memory where the data
for the lattice starts on each processor. Can be found using the {\tt GaugeField()} member function of the {\tt lattice} base class in the Physics system. The number should be entered in hexadecimal without any preceeding 0x.
\item{\bf Lattice Output Filename \ } -- The filename where the lattice
configuration is to be dumped. Be aware of the the file name suffix convention.
\begin{description} 
\item{ \tt .dsp.full \ } Gauge field is stored  in DSP internal floating point representation. All three rows of the link matrices are stored in the file.
\item{ \tt .dsp  \ } Gauge field  is stored in DSP internal floating point representation. The link matrices are stored in compressed 2 row format. The third row is removed.
\item{ \tt .ieee.full \ } Gauge field is stored in IEEE 32Bit Big Endian representation. All three rows of the link matrices are stored.
\item{ \tt .ieee \ } Gauge fields are stored in IEEE 32Bit Big Endian format. The link matrices are stored in compressed 2 row format. The third row is removed.
\end{description}
\item{\bf Ensemble ID \ } -- This is some label uniquely identifying the simulation. It can be any string. On the Gauge Connection, this label is displayed
when your ensemble is listed. By convention configurations from Columbia 
have ensemble ID's of the form {\tt CU\_XXXX } where {\tt CU} identifies
Columbia University (MILC have {\tt MILC} and Ohio State University use 
{\tt OSU}) and {\tt XXXX} is a unique number identifying the ensemble from 
which the gauge is from.
\item{\bf Ensemble Label \ } -- This should be a more human readable label
giving some details about the ensemble such as the parameters that
were used to generate it. This entry takes a string value.
\item{\bf Sequence Number \ } -- This is an integer value identifying 
the configuration in a given ensemble (for example an HMC trajectory number,
or a Heat bath sweep number).
\item{\bf Creator String \ } -- A string describing the group responsible
for the generation of the gauge configuration. Columbia uses {\tt Columbia}.
\item{\bf Creator Hardware \ } -- The machine used to create the lattices.
If the configurations were generated by QCDSP, this field should either be
{\tt CU-QCDSP} for Columbia or {\tt RBRC-QCDSP} for the RIKEN--Brookhaven
machine. 
\item{\bf Datatype \ } -- An entry to describe how to save the configuration.
This can take one of the following values
\begin{description}
\item{\tt 4D\_SU3\_GAUGE \ } Gauge field is to be saved in compressed 2 row
format.
\item{\tt 4D\_SU3\_GAUGE\_3x3 \ } Gauge field is to be saved in full 3 row format.
\end{description}
\item{\bf Floating point type \ } -- Specifies whether the gauge should be
saved in DSP native format or in IEEE format.This can take values:
\begin{description}
\item{\tt TIDSP32 \ } Gauge configuration to be saved in Texas Instruments
DSP 32bit format.
\item{\tt IEEE32BIG \ } Gauge configuration is to be saved in 32 bit IEEE
Big Endian format.
\end{description}
\item{\bf Creation Date \ } -- A character string specifying the date of 
creation. It entry has the value {\tt now} it will be timestamped automatically
when {\tt qunload\_gauge} is invoked.
\end{description}

For example to dump the gauge we have just loaded in the previous section
I would need the following file: 
\begin{verbatim}
# size in T
8

# size in X
8

# size in Y
8

# size in Z
8

# boundary condition for T
ANTIPERIODIC

# boundary conditionf for X
PERIODIC

# boundary condition for Y
PERIODIC

# boundary condition for Z
PERIODIC

# lattice base address
05ed00

# lattice output file
lattice.ieee.full

# Ensemble ID
BALINT_LAT_1

# Ensemble label
Balint's first test lattice 8x8x8x8 pure gauge beta=6.0

# Sequence number 
100

# Creator string 
Columbia

# Creator Hardware 
QCDSP 

# Datatype 
4D_SU3_GAUGE_3x3

# Floating point type
IEEE32BIG

# Creation Date 
now
\end{verbatim}

Before you can use {\tt qunload\_lattice} to unload a configuration
you must tell it about the lattice. If you have created a file such as
above you can tell {\tt qunload\_lattice} to use it as a desctiption file
with the QC-Shell command:
\begin{verbatim}
qset_unload_lattice f <file>
\end{verbatim}
where {\tt <file>} is the name of your description file.

Alternatively you can enter all the fields manually from the command 
line by typing:
\begin{verbatim}
qset_unload_lattice m
\end{verbatim}
and enter answers to the various prompts.

There is a default mode for {\tt qunload\_lattice} which you 
can set with the command:
\begin{verbatim}
qset_unload_lattice d
\end{verbatim}
The default settings however refer to a lattice of $2^4$ sites with 
periodic boundaries in each direction. Hence the default mode is probably
not very useful.

Once you have executed one of the {\tt qset\_unload\_lattice} commands
and have got no error messages, you are ready to unload the lattice.
It is a simple matter of typing 
\begin{verbatim}
qunload\_lattice
\end{verbatim}

For example, I now try to unload the unit gauge lattice I can produce
by running my gauge reading program with no arguments.  I use the example 
lattice description I have listed above except I've changed the output filename
to {\tt unit\_gauge\_8888.ieee.full}.  The description is in a file I call {\tt
qunload.in}.

I first run the program without arguments to produce the unit gauge
\begin{verbatim}
(qcdhost/homeqs0/bj/QCDSP/sfw/hello_world: qcsh[q_1])% qrun simple_test.out 
\end{verbatim}
I get the response
\begin{verbatim}
Local Lattice Size is: 2 4 4 2
Boss: Startup Gauge Field    --  Sum Tr Plaq = 1.000000
Local Gauge field starts at 137c8
\end{verbatim}

I edit my description file {\tt qunload.in} to change the base address to 137C8.
I execute the commands: {\scriptsize
\begin{verbatim}
(qcdhost/homeqs0/bj/QCDSP/sfw/hello_world: qcsh[q_1])% qset_unload_lattice f qunload.in
(qcdhost/homeqs0/bj/QCDSP/sfw/hello_world: qcsh[q_1])% qunload_lattice
\end{verbatim}}
To which the QCDSP replies:
{\scriptsize
\begin{verbatim}
DoUnloadLattice:
        Unloading 4608 words per node from address 0x137c8
        Total transfer of 294912 words from 64 nodes of QCDSP
        Starting to write out 294912 words to host disk
        Finished writing out 1/4 of lattice
        Finished writing out 2/4 of lattice
        Finished writing out 3/4 of lattice
        Finished writing out 4/4 of lattice
        Loading plaq.qin to calculate plaq and trace on QCDSP

Output from plaq.qin running on QCDSP
        Machine size (X,Y,Z,T) (4,2,2,4)
        Lattice size per node (2,4,4,2)
        Total lattice size (8,8,8,8)
        Lattice address 0x137c8

        plaq            1.0000000
        link trace      1.0000000

MkerExecAck:  starting to check program exit status
AnalyzeQCDSP::SCUCheck:  SCUDebug mode 0 doesn't allow analysis
MkerExecAck:  SCUCheck could not be run

DoUnloadLattice:
        copy of header written to lattice.ieee.full

        BEGIN_HEADER
        CHECKSUM =        0
        LINK_TRACE = +1.0000000
        PLAQUETTE = +1.0000000
        DATATYPE = 4D_SU3_GAUGE_3x3
        HDR_VERSION = 1.0
        STORAGE_FORMAT = 1.0
        DIMENSION_1 = 8
        DIMENSION_2 = 8
        DIMENSION_3 = 8
        DIMENSION_4 = 8
        BOUNDARY_1 = PERIODIC
        BOUNDARY_2 = PERIODIC
        BOUNDARY_3 = PERIODIC
        BOUNDARY_4 = ANTIPERIODIC
        ENSEMBLE_ID = BALINT_LAT_1
        ENSEMBLE_LABEL = Balint's first test lattice 8x8x8x8 pure gauge beta=6.0
        SEQUENCE_NUMBER = 100
        CREATOR = Columbia
        CREATOR_HARDWARE = QCDSP 
        CREATION_DATE = Fri Apr 28 18:29:07 2000
        ARCHIVE_DATE = Fri Apr 28 18:29:07 2000
        FLOATING_POINT = IEEE32BIG
        END_HEADER
\end{verbatim}
}

\section{Serial I/O}
We have spent some time dealing with parallel I/O. It should of course
be obvious, that since saving and loading of parallel data is not done
from the user program, that the user program must stop its computation
when parallel IO needs to be performed. This leads to the model of running
Monte Carlo simulations where observables are computed on the fly rather
than the one where one just produces configurations, archives them 
and measures observables later, although this latter model could be 
uses if driven by a shell script say.

Naturally after computing observables on the fly the user would wish to 
store them in a file on the front end. Generally (except for propagators
and things) observables tend to be small objects that can fit into the 
memory of a single node. 

The QCDSP provides the standard C file access mechanism for reading and 
writing to files. {\tt The C++ iostream class is not implemented.}
Hence one can use {\tt fprintf} and {\tt fscanf} to read from a file
and {\tt printf} and {\tt scanf} to read from a terminal. (Bob are the 
binary file functions {\tt fread} and {\tt fwrite} implemented?)

There are however several things that the user has to be aware of:
\begin{itemize}
\item
There is no concept of {\tt stdin}, {\tt stdout} and {\tt stderr} streams
on the QCDSP (although there is on the front end). These are UNIX concepts.
On the QCDSP one can either write to the screen or to a file.
\item
The {\tt printf} command will output on all the processors. The output
of processor 0 is sent to the terminal. The output of other processors
is buffered and can be retreived using the {\tt qprintf} command.
\item
{\bf Access to files on the front end is currently available from node 0 
only. If other nodes wish to read or write files. They must arrange for
node 0 to do it and transfer the results.}
\item
One can turn off I/O by running a program with the command {\tt qrun\_no\_io}.
In this case I/O produced by {\tt printf} and {\tt fprintf} commands is lost.
\end{itemize}

\section{Summary}
In this section we have discussed the main parallel and serial I/O 
capabilities of the QCDSP. We now summarise some of the chief results
of this chapter.

\subsection{Memory Summary} 
User code should reside between addresses 001000 and 05FFFF. The memory
layout of a compiled and linked code is given by the appropriate map file.
This file also includes the heap area so the highest used address in the 
map file will be the highest address used by the program as there will
be no runtime relocation of symbols, since the linker has already resolved
everything.

\subsection{Parallel I/O General Summary} 
Parallel I/O has to be performed before execution of user programs
or after the completion of user programs. Parallel input involves placing
data directly into the QCDSP node memory above the user code but 
below the operating system. Parallel output proceeds by reading directly
the memory of the nodes after a program has stopped executing. 

Parallel output can proceed via the {\tt qread} command for NTFs
or via the {\tt qunload\_lattice} command for lattice gauge fields.

Parallel input can proceed via the {\tt qload} command for NFTs
or via the {\tt qload\_lattice} command for lattice gauge fields in the
Gauge Connection archive format. 

\subsection{Summary of {\tt qread}}
The basic command is {\tt qread <address> <no of blocks>} which will
cause qread to read {\tt <no of blocks>} data words to be read from
each processors memory starting at address {\tt <address>}. Both {\tt
<address>} and {\tt <no of blocks>} are hexadecimal numbers without
any leading 0x.  The behaviour of {\tt qread} can be controlled by the following
QC-Shell Commands
\begin{description}
\item{\tt qset\_read\_output\_filename : \ } Set name of file to dump to.
If no arguments are supplied the command prints the currently set name
of the dump file. If the command is run with one argument it must be
the name of the dump file. Bear in mind that there is a file suffix
convention (see figure \ref{f:NTFTypeSummary}.
\item{\tt qset\_read\_output\_file\_access : \ } Set the access mode of the
dump file. If called with no arguments the currently set mode is displayed.
If called with one argument the argument must be a valid access more. 
Valid access modes are {\tt w} for write mode and {\tt a} for append mode.
\item{\tt qset\_read\_output\_select : \ }
Select output to which the dump will be sent. Dump output can be sent to 
the screen or to a file. If called with no arguments the command displays
the current setting. If called with 1 argument, the argument has to be
either {\tt f} or {\tt s} for dumping to a file or screen respectively.
\item{\tt qset\_read\_output\_tagged : \ }
Select the tagging mode for the dump if output is going to a file (Node Tagged File). Valid tagging modes are 
\begin{description}
\item{\tt no} Do not tag nodes.
\item{\tt yes} Use default node tagging mode (tree)
\item{\tt tree} Use tree node tagging mode
\item{\tt p4d} Use physics (re-mappable) 4D coordinate tagging mode.
\item{\tt m4d} Use machine (non--remappable) 4D coordinate tagging mode.
\end{description}
\end{description}

\subsection{Summary of {\tt qload}}
The {\tt qload} command can be used either to load NTFs or to poke
individual words into memory. It can be invoked in three separate modes
\begin{description}
\item{\tt qload <address> <data> \ } -- loads {\tt <data> } to address
{\tt <address>} on all processors. Both {\tt <data>} and {\tt <address>} 
are hexadecimal values without any preceeding 0x in the notation.
\item{\tt qload <file> \ } -- will cause the loading of the NTF {\tt <file>}.
The memory addresses are hardwired into the NTF so manual relocation may 
be necessary.
\item{\tt qload \ } -- will cause {\tt qload} to execute according to the 
behaviour set with previous commands. Commands affecting {\tt qload} are
\begin{description}
\item{\tt qset\_load\_select \ } Takes 0 or 1  arguments. Selects whether {\tt
qload} operates in screen mode (poking a single data into an address)
or file mode (reading an NFT). When called without an argument, the 
command will print the current selection. The mode will be altered by 
calling {\tt qload} with multiple arguments to load files or data directly.
\item{\tt qset\_load\_screen \ } Can be called with 2 or no arguments.
If called with two arguments, argument 1 must be a hexadecimal address
and argument 2 must be a hexadecimal data value. A subsequent call to 
{\tt qload} will load the data value to the address on all processors if
screen mode has been selected with {\tt qset\_load\_select}, unless
the behaviour of {\tt qload} is altered by some other command.
\item{\tt qset\_load\_filename \ } Can be called with 1 or no arguments.
If called with 1 argument, the argument must be a filename. A subsequent
call to {\tt qload} will attempt to load the specified file if {\tt qload}
has been put into file mode using {\tt qset\_load\_select}, unless the 
behaviour of {\tt qload} has been altered by some other subsequent command.
\end{description}
\end{description}
XS
\subsection{Node Tag File (NTF) Summary}
Node tag files (NTFs)are ASCII files containing data from all the nodes of the
currently selected set of processors. A given NTF contains a magic number
and a sequence of records, one for each processor. Each record contains
some data to identify its processor, the starting address and the number
of blocks of that processors data followed by the data itself. The record
is terminated by a 0. Each entry in an NTF needs to be on a new line.
All numbers are in hexadecimal. For a summary of Node Tag Files see
figures \ref{f:NTFStructure}, \ref{f:NTFRecStructure} and \ref{f:NTFTypeSummary}.

\subsection{Summary of {\tt qload\_lattice}}
The command {\tt qload\_lattice} takes two arguments. Argument one
is the address on each processor where the first word of lattice data
is to reside. The second argument is the name of a lattice gauge configuration
file in the Gauge Connection archive format. 

Executing the command will cause {\tt qload\_lattice} to distribute the 
lattice amongst the currently selected nodes. Any conversion from 
IEEE floating point representation to DSP representation is done at this time.
Various diagnostic information is also printed to the screen, including the
value of the plaquette and the average of the traces of the gauge field 
link matrices.

\subsection{Summary of {\tt qunload\_lattice}}
The command {\tt qunload\_lattice} takes no arguments. Details of the lattice
have to be made known to it previously using the {\tt qset\_unload\_lattice} 
command.

The {\tt qset\_unload\_lattice} command can take 0, 1 or 2 arguments.
If {\tt qset\_unload\_lattice} is called with no arguments it will display
its current settings. If called with 1 argument the argument has to be 
either {\tt d} in which case the default settings will be chosen or {\tt m}
in which case the user will be prompted to enter details manually. If called
with 2 arguments the first argument has to be an {\tt f} followed by the name
of a file containing a description of the lattice to be dumped. This file
must be in the format described in section \ref{s:GaugeUnload}.

\subsection{Gauge Connection Format Summary}
See {\tt http://qcd.nersc.gov}.

\subsection{Serial I/O Summary}
The C Standard I/O file interface is partially implemented. In particular
{\tt printf}, {\tt scanf}, {\tt fprintf}, {\tt fscanf}, {\tt fopen}, {\tt
fclose}, {\tt fread} and {\tt fwrite} are implemented (I am not sure of the
last 2). Apart from {\tt printf} calling these functions only makes sense
from node 0. The {\tt printf} output of processors are saved in a buffer
and can later be recovered using {\tt qprintf} unless the program was run
with {\tt qrun\_no\_io.} in which case all serial I/O is lost.

\subsection{Miscellany}
In this section we have used rather a lot of the QC-Shell (QOS) built
in commands. It is possibly helpful to know that a brief summary of all 
the QC-Shell commands can be listed using the QC-Shell command {\tt qhelp}.
It is recommended that the output is piped through a pager such as {\tt less}
or {\tt more}.

