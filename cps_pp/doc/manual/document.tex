

%------------------------------------------------------------------
% $Id: document.tex,v 1.2 2003-07-30 16:31:42 zs Exp $
%------------------------------------------------------------------
%Anj: EPCC: e-mail: a.jackson@epcc.ed.ac.uk
%
% For best results, this latex file should be compiled using pdflatex.
% However it will also compile under normal latex, if you prefer.
%
%------------------------------------------------------------------
\documentclass[12pt]{article}

% importing other useful packages:
\usepackage{times}
\usepackage{fullpage}
\usepackage{graphicx}
\usepackage{longtable}
\usepackage{tabularx}
% color for the pdf links:
\usepackage{color}
\definecolor{darkblue}{rgb}{0.0,0.0,0.5}
% for conditional latex source:
\usepackage{ifthen}
% pdftex specifications, these are only included if we are using pdflatex:
\providecommand{\pdfoutput}{0}
\ifthenelse{\pdfoutput = 0}{
% Not PDF:
\usepackage{html}
\newcommand{\href}[2]{\htmladdnormallink{#2}{#1}}
}{
% PDF: hyperref for pdf with full linkages:
\usepackage[
pagebackref,
hyperindex,
hyperfigures,
colorlinks,
linkcolor=darkblue,
citecolor=darkblue,
pagecolor=darkblue,
urlcolor=blue,
%bookmarksopen,
pdfpagemode=None,
%=UseOutlines,
pdfhighlight={/I},
pdftitle={The QCDIO API. $Revision: 1.2 $ - $Date: 2003-07-30 16:31:42 $.},
pdfauthor={A.N. Jackson \& S. Booth},
plainpages=false
]{hyperref}
}


% Code style commands:
\newcommand{\cls}[1]{{\bf #1}}            % Classes
\newcommand{\struct}[1]{{\bf #1}}         % Structs
\newcommand{\cde}[1]{{\tt #1}}            % Code fragments

% document style modifications:
\setlength{\parskip}{2.0mm}
\setlength{\parindent}{0mm}
\begin{document}

\title{\bf Physics Environment}
\vskip 1. truein
\author{QCDSP Group}
\maketitle

%\begin{abstract}
%
%\end{abstract}

\tableofcontents

\newpage

%-----------------------------------------------------------
\section{Introduction}
\label{sec-introduction}




%-----------------------------------------------------------
\section{Global Variables}
\label{sec-global}

Objects with the following names are declared external in the corresponding
header files. These objects are defined outside main.


\medskip
{\noindent}
{\bf GJP} Global Job Parameter is a class that
contains all globally needed parameters such as lattice size, verbose
level etc.
\hfill \break
Header File: util/include/global\_job\_parameters.h
\hfill \break
Source Directory: util/global\_job\_parameter


\medskip
{\noindent}
{\bf VRB} VeRBose is a class that controls all verbose output.
\hfill \break
Header File: util/include/verbose.h
\hfill \break
Source Directory: util/verbose


\medskip
{\noindent}
{\bf ERR} Error is a class that controls all error output.
It prints a message and exits with a given exit value.
\hfill \break
Header File: util/include/error.h
\hfill \break
Source Directory: util/error



%-----------------------------------------------------------
\section{Coding Style}
\begin{itemize}

\item There should be no explicit use of the {\tt float} or {\tt double}
native types in the code.  The {\tt Float} typedef should be used instead,
thus allowing compile-time precision control.

\item One may use slightly more modern c++ techniques (template, standard template
library), as long as this code is distinct from the Columbia code, and does not affect up the compilation on the qcdsp.

\end{itemize}

%-----------------------------------------------------------
\section{Internal Data Formats}

\subsection{Lattice: The Gauge Configuration}
Each instance of the \href{../doxygen/html/class_Lattice.html}{Lattice} 
class contains a pointer to the guage
configuration, \cde{gauge\_field}, which is a pointer to an array of
\href{../doxygen/html/class_Matrix.html}{Matrix} objects.  
The value of this pointer can be accessed using the
\href{../doxygen/html/class_Lattice.html#a11}{Lattice::GaugeField()} method.

While the internal format of the guage configuration can be changed, the
"canonical" format will be described here.  The array consists of
4*GPJ.VolNodesSites() Matrix objects, i.e. there four SU(3) matricies
(0, 1, 2, 3 for Ux, Uy, Uz, Ut) for each site on that processor node.
This is the ``fastest-moving index''.  The next is the count in the
x-direction, then the y-direction, then z and finally t.  This detail is
hidden from the user via a simple method: 
\href{../doxygen/html/class_Lattice.html#a13}{Lattice::GsiteOffset(int* x)}.
Given an integer array indicating the \emph{local} position of a site (such
that x[i] is the ith coordinate where i = {0,1,2,3} = {x,y,z,t}), this method
returns a pointer to the first of the four matrix objects associated with that
site.

Each Matrix object is a single SU(3) matrix, build of 2$\times$3$\times$3 Floats 
(i.e. two floats to one complex number, and three by three colors).  [It is
not clear what the internal structure of the SU(3) matrix is, but presumably
the rows and colums of this matrix must map onto the rows and colums of the
matricies read from other codes if binary reproducibility is to be ensured.  -
SHOULD CHECK THIS - ANJ ]. [UPDATE - ADDING A TRANSPOSE FLAG]

\section{Introduction}
The aim of this document is to put forward a plan for the way in which
currently available lattice QCD codes can be adapted to run on the
forthcoming QCD-OC hardware.  This will focus on the C++ QCD-SP code,
and in particular on how this code may be made a portable as possible,
so that a minimum of new code will be required for it to run on
any future hardware platform.

\subsection*{The Core Aim:}
\begin{quote}
To develop the codes required to produce good science using the QCD-OC
machine.
\end{quote}

The quality of the science that can be determined using the code is
predominately determined by the functionality supplied by it.  This is
primary issue is outlined in \S\ref{codefunc} below.  Having defined
the functionality we require, we can then consider the best route by
which to produce such a code.  There is a large volume of trusted code
already in existence, and so this secondary issue breaks down as
follows.  We evaluate the available codes, estimating the pro's and
con's of each (see \S\ref{oldcodes}), and then put forward plans for
ways in which these code may be developed to meet our aims
(\S\ref{plans}).

\subsection*{Overall Constraints}
\begin{itemize}
\item The most important constraint is the project time-scale.  The
hardware may be available in as little as 12 months, and the software
development plan should fit \emph{within} this ETA.
\end{itemize}

%--------------------------------------------------------------------
\section{Functionality Specification for the QCD-OC code}
\label{codefunc}
The follow specification is a breakdown of the \emph{ideal}
functionality that the new code could provide.  A version of the code
which contains all of the assorted bells \& whistles listed below is
probably unattainable on the project time-scale.  However, all HIGH
priority functionality should be included, and the code-developers
will keep in mind the possible implementation of the lower-priority
aims at a later date.

%---------------------------
\subsection{The Physical System}
\begin{center}
 \begin{tabular}{lc}
 \hline
 \bf{Description} & \bf{Priority}\\
 \hline
  Wilson gauge field & HIGH \\
  Wilson-clover fermions & HIGH \\
  Ginsburg-Wilson fermion action & Medium (long-term HIGH)\\
 \hline
 \end{tabular}
\end{center}

%---------------------------
\subsection{Evolution Mechanisms}
\begin{center}
 \begin{tabular}{lc}
 \hline
 \bf{Description} & \bf{Priority}\\
 \hline
 Hybrid MC. & HIGH \\
 Tweakability of algorithm parameters. & HIGH \\
 Output options? & ? \\
 \hline
 \end{tabular}
\end{center}

%---------------------------
\subsection{Measurement Capabilities}
\begin{center}
 \begin{tabular}{lc}
 \hline
 \bf{Description} & \bf{Priority}\\
 \hline
  $\bar{\psi}\psi$? & ?\\
 \hline
 \end{tabular}
\end{center}

%---------------------------
\subsection{Constraints \& Overall Priorities}
\begin{center}
 \begin{tabular}{lc}
 \hline
 \bf{Description} & \bf{Priority}\\
 \hline
  Usable within $\sim$12 months. & HIGH \\
  Capable of using 5x5x5x5 sublattices, i.e. the odd local lattice
 size issue. &HIGH \\
  Run-time machine partitioning in software & Medium \\
 Portability and future proofing against future hardware. & Medium/low \\
 \hline
 \end{tabular}
\end{center}

%------------------------------------------------------------------
\section{Available Codes}
\label{oldcodes}

\subsection{Colombia/QCD-SP}
Written in pre-ANSI C++.  What do to about this? (what are Colombia
going to do?)  See \S\ref{col:notcpp}. But review of this code in
\S\ref{col:code}. 

\subsection{SZIN}
This is a macro-based application code, and as such its treatment is
split into two parts.  Firstly, the definition of the language itself,
and secondly the application code that has been built up using that language.

\subsubsection{SZIN: The Language }
[m4 macros? links?]

\begin{itemize}
\item In Brief:
	\href{http://www.jlab.org/~edwards/szin_manual.html}
	{http://www.jlab.org/{\textasciitilde}edwards/szin\_manual.html}
\item Manual:
	\href{http://www.jlab.org/~edwards/macros_v2.ps}
	{http://www.jlab.org/{\textasciitilde}edwards/macros\_v2.ps}
\end{itemize}

\subsubsection{SZIN: The Application Code}

\subsection{UKQCD}
[What fortran codes are availible, and are there any ways in which
this code may be of use to use despite the (probable???) lack of a
compiler.] 

%------------------------------------------------------------------
\section{Overview of the Colombia/QCD-SP code}
\label{col:code}
The core concept behing this implementation of a lattice QCD library
is the unification of the way in which different gauge and fermion
actions are treated, via \cde{lattice} abstract class
(\S\ref{col:latticeclass}).  The algorithms to be applied to the
chosen system are also abstracted in order to unify the interface by
which algorithms are `run' on the system (\S\ref{col:algclass}).
These classes, along with simple mechanisms for specifing the simulation
parameters, allow the user to construct LQCD applications using
relatively little source code.  A simple source file can be
constructed as follows:
\begin{itemize}
\item Specify the global parameters, such as the size of the
processor array (\S\ref{col:globalparam}).
\item Define the system by choosing the gauge and fermion types
(\S\ref{col:latticeclass}).
\item Choose an algorithm from the suite available
(\S\ref{col:algclass}).
\item Specify any parameters specific to the algorithm (\S\ref{col:algparam}).
\item Run the algorithm for a given number of steps (\S\ref{col:algclass}).
\end{itemize}

The central aim of the following code breakdown is twofold:
\begin{enumerate}
  \item To clarify the library interface, so that any shortcomings in
  the way in which the application interacts with the library are made
  clear.
  \item To indicate any gaps between the desired code functionality
  and that supplied by this code, particularly where the gaps are due
  to QCD-SP specific parts of the code.
\end{enumerate}

[???, What to say about sub-lattice size limitations???]
[???, What to say about software partiioning issue???]

%---------------------------
\subsection{Global Parameters}
\label{col:globalparam}
These either specify some property of the simulation itself
(\S\ref{col:gjp} \& \S\ref{col:commarg}), or control the textual output
of the code (\S\ref{col:verboseclass} \& \S\ref{col:errorclass}).

\subsubsection{The GlobalJobParameter class and the DoArg \& CommonArg structs}
\label{col:gjp}
To control the most general parameters of the simulation, a specific  instance
of \cls{GlobalJobParameter} class (called \cde{GJP} and with global scope)
\emph{must} be created. The parameters contained in the \cde{GJP} can
be read via a set of methods, but these parameters should be set by
using a single call and a \struct{DoArg} struct.

The parameters held in an instance of \struct{DoArg} are listed in full
in table \ref{tab:doargs}.  In brief, this allows the user to set:
\begin{itemize}
 \item The parameters of the regular domain decomposition, i.e. the
 size of the processor grid and the size of the lattice on each
 processor.
 \item The boundary conditions (periodic or antiperiodic in each
 direction).
 \item The kind of initial configuration.
 \item The initial value of the random number generator seed.
 \item The number of colors.
 \item Various parameters for controlling the different gauge/fermion actions,
 such as the value of the gauge $\beta$.
\end{itemize}
The parameters held within a \struct{DoArg} structure are then used to
initialise the global job parameters held in \cde{GJP}.

\subsubsection{The CommonArg struct}
\label{col:commarg}
This structure is used to control the output of the code, and an
instance of it is used to initialise each of the algorithms
(\S\ref{col:algclass}).  It contains a unique simulation identifier
and the name of the file to be used for simulation output.
[???, The unique ID is not used in the test codes, it seems.  Does
this have something to do with the task list and job manager stuff???]

\subsubsection{Verbosity: The Verbose class}
\label{col:verboseclass}
As in the case of the \cls{GlobalJobParameter} class, a specific
instance of the \cls{Verbose} class (called \cde{VRB}) is used to control
the diagnostic textual output of the code.  This information includes
subroutine call traces, memory allocation and deallocation and so on.
It also allows the program to control the LED status.  The level of
diagnostic output is initially set via the \cls{GlobalJobParameter}
initialisation, but can be explicitly set using \cde{VRB.Level}.

\subsubsection{Error Reporting: The Error class}
\label{col:errorclass}
Again, a specific global instance of the \cls{Error} class (called
\cde{ERR}) is used to control the error reporting mechanism.  This is
very similar to the \cls{Verbose} class, but for serious problems
(such as attempts to use \cde{NULL} pointers, file I/O failures and so
on) which require the code to cease execution.

%---------------------------
\subsection{The Lattice Class}
\label{col:latticeclass}
This class forms the core of the QCDSP code.  It holds the gauge-field
configuration, and defines the interface for the gauge-field and
fermion-field operations in such a way as to ensure that the interface
remains the same no matter what specific action is used.  It also
defines the interface to the random number generator. [???, anything
else??? Converting???]
Note that the fermion field is actually stored elsewhere (usually in
the \cls{Alg} class, \S\ref{col:algclass}). A large number of classes
are derived from this base class, for each fermion and gauge action.
The actual class that is used for a simulation will be one of these
derived classes, chosen by the dessired combination of actions.

\subsubsection{Gauge Field Types}
There are four possible gauge-field types (see table
\ref{tab:gauges}), each of which implements the gauge-operation
interface specified by the \cls{Lattice} class (see
\S\ref{sss:gaugeops}).

\begin{table}[!ht]
\begin{center}
 \begin{tabular}{lp{100mm}l}
 \hline
 \bf{Class} & \bf{Description} & \bf{Dependencies}\\
 \hline
 Gnone & Acts as if there is no gauge action (i.e. $\beta=0$) & ???\\
 Gwilson & Uses the standard Wilson single plaquette action. & ???,The
 Wilson Library?\\
 GimprRect & Uses the standard Wilson plaquette operator plus
the second order rectangle operator. ???& \\
 GpowerPlaq & This action is the same as the standard Wilson action
with the irrelevant power plaquette term added to it. The full action
is: $\sum_p [ \beta * { -Tr[U_p]/3} + ( {1 - Tr[U_p]/3} / c )^k ]$
with \cde{c = GJP.PowerPlaqCutoff()} and \cde{k =
GJP.PowerPlaqExponent()}.  This action supresses plaquettes with ${1 -
ReTr[U_p]/3} > c$ and threfore reduces lattice dislocations. & ???\\
 GpowerRect & Uses the standard Wilson plaquette operator plus the
second order rectangle operator plus a power plaquette term plus a
power rectangle term.  The full action is: $( \sum_p [ c_0*\beta*
{ -Tr[U_p]/3} + ( {1 - Tr[U_p]/3} / c )^k ] + \sum_r [ c_1*\beta*
{ -Tr[U_r]/3} + ( {1 - Tr[U_r]/3} / c )^k ] )$ with \cde{c =
GJP.PowerPlaqCutoff()}, \cde{k = GJP.PowerPlaqExponent()} \cde{c\_0 = 1
- 8 * c\_1}, \cde{c\_1 = GJP.C1()}. This action supresses plaquettes
with ${1 - ReTr[U_p]/3} > c$ and rectangles with ${1 - ReTr[U_r]/3} >
c$ and therefore reduces lattice dislocations.& ???\\
 \hline
 \end{tabular}
 \label{tab:gauges}
 \caption{Gauge fields supported by the QCD-SP code.}
\end{center}
\end{table}

\subsubsection{Gauge Operations Interface}
\label{sss:gaugeops}
[What functionality for acting upon the Lattices is supplied within Lattice.]
\begin{itemize}
\item 
{\bf Gclass\-Type} {\bf Gclass} (void)
\begin{list}\small\item\em It returns the type of gauge
class.\item\end{list}

\item 
void {\bf Gaction\-Gradient} ({\bf Matrix} \&grad, int $\ast$x, int
mu)
\begin{list}\small\item\em Calculates the partial derivative of the gauge action w.r.t. the link U\_\-mu(x). Typical implementation has this func called with \cls{Matrix} \&grad = $\ast$mp0, so avoid using it. \item\end{list}

\item 
void {\bf Gforce\-Site} ({\bf Matrix} \&force, int $\ast$x, int mu)
\begin{list}\small\item\em It calculates the gauge force at site x and
direction mu.\item\end{list}

\item 
void {\bf Evolve\-Mom\-Gforce} ({\bf Matrix} $\ast$mom, {\bf Float} step\_\-size)
\begin{list}\small\item\em It evolves the canonical momentum mom by step\_\-size using the pure gauge force. \item\end{list}

\item 
{\bf Float} {\bf Ghamilton\-Node} (void)
\begin{list}\small\item\em Returns the value of the pure gauge Hamiltonian of the node
sublattice.\item\end{list}

\end{itemize}

[Plus some action specific functions.]

[Details of dependencies.vector\_util, Matrix, Float, cbuf (this last
one turns into nothing on a workstation, I think).]

\subsubsection{Fermion Action Types}

\begin{table}[!ht]
\begin{center}
 \begin{tabular}{lp{100mm}l}
 \hline
 \bf{Class} & \bf{Description} & \bf{Dependencies}\\
 \hline
 Fnone & Its functions do nothing and return values as if there is no
 fermion action or fermion fields. The number of spin components is
 zero & ???\\
 FstagTypes: & Staggered fermions: & \\
 \hspace{5mm}Fstag & Defines the \cls{Lattice} functions for
 staggered fermions. & ???\\
 FwilsonTypes: & Wilson fermions:& \\
 \hspace{5mm}Fclover & Wilson-clover fermions. & ???,The Wilson Library?\\
 \hspace{5mm}Fdwf & Wilson-domain-wall fermions. & ???\\
 \hspace{5mm}Fwilson & Wilson fermions. & ???\\
 \hline
 \end{tabular}
 \label{tab:factions}
 \caption{Fermion actions supported by the QCD-SP code.}
\end{center}
\end{table}

\subsubsection{Fermion Operations Interface}
[What functionality for acting upon the Lattices is supplied within Lattice.]
[NB It contains an interface for a Ritz eigenvec/val solver... Which
is implemented in eigen\_wilson.C.  This is relevant to the GW stuff]

\begin{itemize}

\item 
{\bf Fclass\-Type} {\bf Fclass} (void)
\begin{list}\small\item\em It returns the type of fermion
class.\item\end{list}

\item 
int {\bf Fsite\-Offset\-Chkb} (const int $\ast$x) const
\begin{list}\small\item\em 
Sets the offsets for the fermion fields on a  checkerboard. The
fermion field storage order is not the canonical one but it is
particular to the fermion type. This function is not relevant to
fermion types that do not use even/odd checkerboarding. x[i] is the
ith coordinate where i = \{0,1,2,3\} = \{x,y,z,t\}. 
\item\end{list}

\item 
int {\bf Fsite\-Offset} (const int $\ast$x) const
\begin{list}\small\item\em 
Sets the offsets for the fermion fields on a  checkerboard. The
fermion field storage order is the canonical one. X[I] is the ith
coordinate where i = \{0,1,2,3\} = \{x,y,z,t\}. 
\item\end{list}

\item 
int {\bf Exact\-Flavors} (void)
\begin{list}\small\item\em 
Returns the number of exact flavors of the matrix that is inverted
during a molecular dynamics evolution. 
\item\end{list}

\item 
int {\bf Spin\-Components} (void)
\begin{list}\small\item\em Returns the number of spin
components.\item\end{list}

\item 
int {\bf Fsite\-Size} (void)
\begin{list}\small\item\em 
Returns the number of fermion field  components (including
real/imaginary) on a site of the 4-D lattice.
\item\end{list}

\item 
int {\bf Fchkb\-Evl} (void)
\begin{list}\small\item\em 
0 -$>$ If no checkerboard is used for the evolution or the CG that
inverts the evolution matrix. 1 -$>$ If the fermion fields in the
evolution or the CG that inverts the evolution matrix are defined on a
single checkerboard (half the  lattice). 
\item\end{list}

\item 
int {\bf Fmat\-Evl\-Inv} ({\bf Vector} $\ast$f\_\-out, {\bf Vector} $\ast$f\_\-in, {\bf Cg\-Arg} $\ast$cg\_\-arg, {\bf Float} $\ast$true\_\-res, {\bf Cnv\-Frm\-Type} cnv\_\-frm=CNV\_\-FRM\_\-YES)
\begin{list}\small\item\em
It calculates f\_\-out where A $\ast$ f\_\-out = f\_\-in and A is the
preconditioned (if relevant) fermion matrix that appears in the HMC
evolution (typically some preconditioned  form of [Dirac$^\wedge$dag
Dirac]). The inversion is done with the conjugate gradient. cg\_\-arg
is the structure that contains all the control parameters, f\_\-in is
the fermion field source vector, f\_\-out should be set to be the
initial guess and on return is the solution. f\_\-in and f\_\-out are
defined on a checkerboard. If true\_\-res !=0 the value of the true
residual is returned in true\_\-res. true\_\-res = $|$src -
Mat\-Pc\-Dag\-Mat\-Pc $\ast$ sol$|$ / $|$src$|$ The function returns
the total number of CG iterations.
\item\end{list}

\item 
int {\bf Fmat\-Evl\-Inv} ({\bf Vector} $\ast$f\_\-out, {\bf Vector} $\ast$f\_\-in, {\bf Cg\-Arg} $\ast$cg\_\-arg, {\bf Cnv\-Frm\-Type} cnv\_\-frm=CNV\_\-FRM\_\-YES)
\begin{list}\small\item\em Same as original but with
true\_\-res=0;.\item\end{list}

\item 
int {\bf Fmat\-Inv} ({\bf Vector} $\ast$f\_\-out, {\bf Vector} $\ast$f\_\-in, {\bf Cg\-Arg} $\ast$cg\_\-arg, {\bf Float} $\ast$true\_\-res, {\bf Cnv\-Frm\-Type} cnv\_\-frm=CNV\_\-FRM\_\-YES, {\bf Preserve\-Type} prs\_\-f\_\-in=PRESERVE\_\-YES)
\begin{list}\small\item\em 
It calculates f\_\-out where A $\ast$ f\_\-out = f\_\-in and A is the
fermion matrix (Dirac operator). The inversion is done with the
conjugate gradient. cg\_\-arg is the  structure that contains all the
control parameters, f\_\-in  is the fermion field source vector,
f\_\-out should be set  to be the initial guess and on return is the
solution. f\_\-in and f\_\-out are defined on the whole lattice. If
true\_\-res !=0 the value of the true residual is returned in
true\_\-res. true\_\-res = $|$src - Mat\-Pc\-Dag\-Mat\-Pc $\ast$
sol$|$ / $|$src$|$ cnv\_\-frm is used to specify if f\_\-in should be
converted  from canonical to fermion order and f\_\-out from fermion
to canonical.  prs\_\-f\_\-in is used to specify if the source f\_\-in
should be preserved or not. If not the memory usage is less by the
size of one fermion vector or by the size  of one checkerboard fermion
vector (half a fermion vector). For staggered fermions f\_\-in is
preserved regardles of the value of prs\_\-f\_\-in.  The function
returns the total number of CG iterations. 
\item\end{list}

\item 
int {\bf Fmat\-Inv} ({\bf Vector} $\ast$f\_\-out, {\bf Vector} $\ast$f\_\-in, {\bf Cg\-Arg} $\ast$cg\_\-arg, {\bf Cnv\-Frm\-Type} cnv\_\-frm=CNV\_\-FRM\_\-YES, {\bf Preserve\-Type} prs\_\-f\_\-in=PRESERVE\_\-YES)
\begin{list}\small\item\em Same as original but with
true\_\-res=0;.\item\end{list}

\item 
int {\bf Feig\-Solv} ({\bf Vector} $\ast$$\ast$f\_\-eigenv, {\bf Float} $\ast$lambda, {\bf Float} chirality[$\,$], int valid\_\-eig[$\,$], {\bf Float} $\ast$$\ast$hsum, {\bf Eig\-Arg} $\ast$eig\_\-arg, {\bf Cnv\-Frm\-Type} cnv\_\-frm=CNV\_\-FRM\_\-YES)
\begin{list}\small\item\em 
It finds the eigenvectors and eigenvalues of A where A is the fermion
matrix (Dirac operator). The solution uses Ritz
minimization. eig\_\-arg is the  structure that contains all the
control parameters, f\_\-eigenv are the fermion field source vectors
which should be defined initially, lambda are the eigenvalues returned
on solution. f\_\-eigenv is defined on the whole lattice. hsum are
projected eigenvectors. The function returns the total number of Ritz
iterations. 
\item\end{list}

\item 
void {\bf Set\-Phi} ({\bf Vector} $\ast$phi, {\bf Vector} $\ast$frm1, {\bf Vector} $\ast$frm2, {\bf Float} mass)
\begin{list}\small\item\em It sets the pseudofermion field phi from
frm1, frm2.\item\end{list}

\item 
void {\bf Evolve\-Mom\-Fforce} ({\bf Matrix} $\ast$mom, {\bf Vector} $\ast$frm, {\bf Float} mass, {\bf Float} step\_\-size)
\begin{list}\small\item\em 
It evolves the canonical momentum mom by step\_\-size using the
fermion force. 
\item\end{list}

\item 
{\bf Float} {\bf Fhamilton\-Node} ({\bf Vector} $\ast$phi, {\bf Vector} $\ast$chi)
\begin{list}\small\item\em 
The fermion Hamiltonian of the node sublattice. chi must be the
solution of Cg with source phi. 
\item\end{list}

\item 
void {\bf Fconvert} ({\bf Vector} $\ast$f\_\-field, {\bf Str\-Ord\-Type} to, {\bf Str\-Ord\-Type} from)
\begin{list}\small\item\em Convert fermion field f\_\-field from -$>$
to.\item\end{list}

\item 
{\bf Float} {\bf Bhamilton\-Node} ({\bf Vector} $\ast$boson, {\bf Float} mass)
\begin{list}\small\item\em The boson Hamiltonian of the node
sublattice.\item\end{list}
\end{itemize}

[Plus action specific functions.]

[Details of dependencies.]

\subsubsection{Putting the fermion \& gauge fields together}
[The system is chosen from a list of all possible F and G
combinations: GtypeFtype]

%---------------------------
\subsection{Dirac Operators}

%---------------------------
\subsection{The Algorithms}
\label{col:algclass}

\subsubsection{Algorithm Parameters}
\label{col:algparam}




\end{document}




