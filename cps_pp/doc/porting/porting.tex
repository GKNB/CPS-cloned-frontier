#include<config.h>
CPS_START_NAMESPACE
%------------------------------------------------------------------
% $Id: porting.tex,v 1.1.1.1 2003-06-22 13:34:51 mcneile Exp $
%------------------------------------------------------------------
%Anj: EPCC: e-mail: a.jackson@epcc.ed.ac.uk
%
% For best results, this latex file should be compiled using pdflatex.
% However it will also compile under normal latex, if you prefer.
%
%------------------------------------------------------------------
\documentclass[12pt]{article}

% importing other useful packages:
\usepackage{times}
\usepackage{fullpage}
\usepackage{graphicx}
\usepackage{longtable}
\usepackage{tabularx}
% color for the pdf links:
\usepackage{color}
\definecolor{darkblue}{rgb}{0.0,0.0,0.5}
% for conditional latex source:
\usepackage{ifthen}
% pdftex specifications, these are only included if we are using pdflatex:
\providecommand{\pdfoutput}{0}
\ifthenelse{\pdfoutput = 0}{
% Not PDF:
\usepackage{html}
\newcommand{\href}[2]{\htmladdnormallink{#2}{#1}}
}{
% PDF: hyperref for pdf with full linkages:
\usepackage[
pagebackref,
hyperindex,
hyperfigures,
colorlinks,
linkcolor=darkblue,
citecolor=darkblue,
pagecolor=darkblue,
urlcolor=blue,
%bookmarksopen,
pdfpagemode=None,
%=UseOutlines,
pdfhighlight={/I},
pdftitle={Installing CPS. $Revision: 1.1.1.1 $ - $Date: 2003-06-22 13:34:51 $.},
pdfauthor={A.N. Jackson \& S. Booth},
plainpages=false
]{hyperref}
}


% Code style commands:
\newcommand{\cls}[1]{{\bf #1}}            % Classes
\newcommand{\struct}[1]{{\bf #1}}         % Structs
\newcommand{\cde}[1]{{\tt #1}}            % Code fragments

% document style modifications:
\setlength{\parskip}{2.0mm}
\setlength{\parindent}{0mm}

% Questionbox commands:
\newcounter{quescount}
\setcounter{quescount}{0}
\newcommand{\quesbox}[2]{\begin{center}\refstepcounter{quescount}\fbox{\parbox{130mm}{
\label{#1}{\bf Q. \arabic{quescount}:} #2} } \end{center} }


% title information:
\title{Installing the Columbia Physics System}
\author{A.N. Jackson \& S. Booth}
\date{\mbox{\small $$Revision: 1.1.1.1 $$  $$Date: 2003-06-22 13:34:51 $$}}

%------------------------------------------------------------------
\begin{document}

\maketitle

\begin{flushright}
\href{mailto:A.N.Jackson@ed.ac.uk}{A.N.Jackson@ed.ac.uk}.
\end{flushright}

\tableofcontents

\newpage

%------------------------------------------------------------------
\section{Downloading the CPS source code}

%------------------------------------------------------------------
\section{Compiling the code}

\subsection{INSTALLATION ON QCDSP} 
The original make structure has been retained, and the configuration
files are set up to default to the target being QCDSP.  Therefore, the
code should compile straight away by simply running make in the phys/
directory.


\subsection{INSTALLATION ELSEWHERE}

{\large \emph{You will require gmake (GNU make ) to compile this program, and to
run it in parallel you will require some implementation of MPI.} }

The code configuration is done in config.h. The compilation
configuration is done in Makefile.gnu and Makefile.gnutests.  The
regression-testing configuration is done in tests/regression.pl.

\subsubsection{The configure script}
The code uses a fairly standard configure script for cross-platform
configuration of the makefiles etcetera.  The script should be run from the
phys/ directory as ``./configure''.  There are a large number of general
configuration flags (use ``./configure --help'' to see them all) but only a
few of these influence the behaviour of the physics code itself:
\begin{itemize}
 \item{\tt --enable-parallel-mpi} --- Compile as a parallel code, using MPI. (default=no) 
 \item{\tt --enable-double-prec} --- Force the code to use double precision (instead of float) throughout (default=yes) 
\end{itemize}

If all goes well, you may build via the GNU make files using:
\begin{itemize}
 \item{\tt gmake -f Makefile.gnu [all]} --- Makes everything.
 \item{\tt gmake -f Makefile.gnu clean} --- Cleans everything.
 \item{\tt gmake -f Makefile.gnu testprogs} --- Makes only the test suite.
 \item{\tt gmake -f Makefile.gnu cleantests} --- Cleans only the test suite.
\end{itemize}

\subsubsection{Manual configuration}
The current configure script handles ome things (e.g. the MPI include/linkage)
in a platform (Solaris) specific manner.  Therefore compilation for anything but serial gcc
may require some minor tuning.  The files that you may have to change are as
follows:

\begin{itemize}
\item{\tt config.h}\\

This file contains the following flags:
\begin{itemize}
 \item{\tt PARALLEL} If defined, compile for a parallel (MPI) environment.
 \item{\tt CAST\_AWAY\_CONST}  A macro describing how the current compiler casts
                     away the const-ness of strings (inline const
                     char*).  This is not strictly necessary and you
                     should not have to change it.
 \item{\tt GLOBALSUM\_TYPE} Precision/type for global summations.  On QCDSP
 this was a custon double64 type, but on other platforms double should be
 used.
 \item{\tt LOCALCALC\_TYPE} Precision/type for Floats.  On QCDSP ths was a
 custom float type (rfloat), but a native type should be used elsewhere.
 \item{\tt INTERNAL\_LOCALCALC\_TYPE} Precision/type for IFloat.  Should be the
                             same as LOCALCALC\_TYPE anywhere but on
                             QCDSP, where float should be used (IFloats used
			     to be floats in the original code).
 \item{\tt COMMS\_DATASIZE} Default size of the individual items of data to be
                    transmitted via the MPISCU.  Should be consistent
                    with the size of Float and IFloat.
 \item{\tt VERBOSE} If defined, the MPISCU will print information concerning
             each of the SCU calls.
\end{itemize}

The \#include of qcdio.h means that printf is overridden with qprintf,
which is an implementation of printf designed to print only the
information from the zeroth node.  This makes the code output consistent
with the default code output on QCDSP.

\item{\tt Makefile.gnu \& Makefile.gnutests}\\
The elements you may have to change
lie between the ***COMPILATION FLAGS*** markers.  

In Makefile.gnu, this concerns the compiler name and flags (the
{\tt -I(top\_dir)/nga/mpi\_scu} tells the code where the MPISCU include files
are).

In Makefile.gnutests, you must choose which set of library files to
compile against (depending on whether the code is serial or
MPI-parallel) and define the compilation and linking flags.

\item{Testing Script {\tt tests/regression.pl}}\\
At the start of tests/regression.pl three different examples of
testing configuration are given.  The first set runs the parallel
QCDSP version, the second set runs the serial QCDSP version and the
third set runs the non-QCDSP version.  

\end{itemize}



%-------------------------------------------------------------------
\section{Platforms}
The code has been used on the following platforms.

\subsection{Columbia: QCDSP}
The serial and parallel versions compile straight out of the box using
the "gmake -f Makefile.gnu" and "make" commands respectively.

\subsection{EPCC: Cray T3E}
For compilation the following set of commands seems to work:
\begin{verbatim}
bash$ export PATH=/opt/open/bin:$PATH
bash$ autoconf
bash$ ./configure --enable-parallel-mpi CC=CC CFLAGS="-O3 -h conform"
bash$ make -f Makefile.gnu all
\end{verbatim}

I don't know why you have to run autoconf on the T3E.  It must be some
localised abberation because it is not necessary on other platforms.

\subsection{UKQCD: Ukqcd2; an Alpha}

\subsection{EPCC: Bronzite; a Sun workstation}

\subsection{EPCC: Lomond; a Sunfire 6800 SMP}

\subsection{EPCC: Bobcat; a Beowulf cluster}


\end{document}



CPS_END_NAMESPACE
