

%------------------------------------------------------------------
% $Id: testing.tex,v 1.4 2003-08-08 13:29:39 mcneile Exp $
%------------------------------------------------------------------
%Anj: EPCC: e-mail: a.jackson@epcc.ed.ac.uk
%
% For best results, this latex file should be compiled using pdflatex.
% However it will also compile under normal latex, if you prefer.
%
%------------------------------------------------------------------
\documentclass[12pt]{article}

% importing other useful packages:
\usepackage{times}
\usepackage{fullpage}
\usepackage{graphicx}
\usepackage{longtable}
\usepackage{tabularx}
% color for the pdf links:
\usepackage{color}
\definecolor{darkblue}{rgb}{0.0,0.0,0.5}
% for conditional latex source:
\usepackage{ifthen}
% pdftex specifications, these are only included if we are using pdflatex:
\providecommand{\pdfoutput}{0}
\ifthenelse{\pdfoutput = 0}{
% Not PDF:
\usepackage{html}
%%\newcommand{\href}[2]{\htmladdnormallink{#2}{#1}}
}{
% PDF: hyperref for pdf with full linkages:
\usepackage[
pagebackref,
hyperindex,
hyperfigures,
colorlinks,
linkcolor=darkblue,
citecolor=darkblue,
pagecolor=darkblue,
urlcolor=blue,
%bookmarksopen,
pdfpagemode=None,
%=UseOutlines,
pdfhighlight={/I},
pdftitle={Frameworks For Testing. $Revision: 1.4 $ - $Date: 2003-08-08 13:29:39 $.},
pdfauthor={A.N. Jackson, C. McNeile \& S. Booth},
plainpages=false
]{hyperref}
}


% Code style commands:
\newcommand{\cls}[1]{{\bf #1}}            % Classes
\newcommand{\struct}[1]{{\bf #1}}         % Structs
\newcommand{\cde}[1]{{\tt #1}}            % Code fragments

% document style modifications:
\setlength{\parskip}{2.0mm}
\setlength{\parindent}{0mm}

% Questionbox commands:
\newcounter{quescount}
\setcounter{quescount}{0}
\newcommand{\quesbox}[2]{\begin{center}\refstepcounter{quescount}\fbox{\parbox{130mm}{
\label{#1}{\bf Q. \arabic{quescount}:} #2} } \end{center} }


% title information:
\title{Testing the Columbia code.}
\author{A.N. Jackson, C. McNeile,  \& S. Booth}
\date{\mbox{\small $$Revision: 1.4 $$  $$Date: 2003-08-08 13:29:39 $$}}

%------------------------------------------------------------------
\begin{document}

\maketitle

\tableofcontents
\newpage

%-------------------------------------------------------------------
\section{Testing Framework for the cps++ code.}

We are ``growing'' a testing framework for the cps++ code.
The original testing framework was written by Chris Miller
at Columbia.


This testing framework is meant to be used for the 
cps++ code on its own with out any additional libraries.
We expect that UKQCD users will interact more with 
QCD-lite supercharged version of cps++, but it is useful to 
have a framework for the cps code on its own.

\subsection{Running the tests}

The tests are run using the makefile target.
\begin{verbatim}
make -f Makefile_cps test
\end{verbatim}
%%
This should produce output like
\begin{verbatim}
Starting test w_spect
make[1]: Entering directory `/raidz/mcneile/qcdoc/cps++/tests/w_spect'
g++ -o regression_cps.x -DPARALLEL=0 -DTARGET_QCDSP=0  -O2
-I../../../cps++ -I../.../cps++/include  main.C
``../../../cps++''/cps.a  -lm
make[1]: Leaving directory `/raidz/mcneile/qcdoc/cps++/tests/w_spect'
w_spect    PASS
\end{verbatim}

A perl script loops over the test cases, compiles the code,
and then runs it. The output is compared against
previous output and a PASS/FAIL tag is generated.

\subsection{Philosophy behind the testing framework}

Lattice QCD codes produce a lot of output. It is important that the 
output is automatically tested against older output, otherwise
a user could miss a problem.

Because of rounding errors, the output can not be tested against older
output by using a crude tool such as diff. We have taken the
philosophy that only an important subset of the data needs to be
tested. For example, a regression test may only need to check that the
pion correlator at a single timeslice is correct. If the pion
correlator at timeslice 2 is correct, then it is likely that all the
meson correlators are correct.

In this testing framework, we use perl scripts to
extract a subset of the correlators. The important
perl script that does the comparison is called
regressions/check\_data.pl. This compares the output from
the program to previous output stored in a special format.

\subsection{Adding a new test to the testing framework}

%-------------------------------------------------------------------
\section{What is required from a  Testing Framework}

Here are some thoughts on what is required for a testing
framework. Most of the information in this section will
be moved elsewhere at some stage.

\subsection{Parallelism Issues}
Some of the test cases should be done on big lattices, as there are subtle communication bugs that will only show with more than 32 nodes.  Therefore, the code should ensure the same results can be calculated over a range of machine partition sizes.

\subsection{Input Parameters}
One of the issues is how to maintain the tests 
as the input parameters to the code gets changed. One way to
proceed would be to write a library of perl scripts to write 
the input parameters to the codes. This would localise most of the 
input parameters to the code in one place. (This approach is 
recommended in Software Test Automation by Fewster and 
Graham).


\subsection{Output Format}
One of problems with testing is comparing the output against the "correct" output. The issue is how to extract the important output (pion correlators, and residues) from the unimportant output (dates).  It is suggested that we use XML to tag the important information. 
For example, the output could look "something like":
\begin{verbatim}
<notimportant>
STATUS:solver:make_source:0:Using point source on spin 0 colour 0
 STATUS:solver:make_clover:0:Making clover for p = 1
 STATUS:solver:make_source:0:Using point source on spin 0 colour 0
 STATUS:solver:solver_g5_driver:0:real residue is  
</notimportant>
<residue> 0.97820832E-09 </residue>
<notimportant>
STATUS:solver:solver_data_save:0:Saving solver data...
 STATUS:solver:solver_mpp_save:0:Saving solver data...
</notimportant>
\end{verbatim}
The plan would be to use "attributes" to index
things such as the residue as well.

An XML parser such as expat (\href{http://www.jclark.com/xml/expat.html}{http://www.jclark.com/xml/expat.html})
could be used to do the comparison.  Another option might be to use xmldiff 
(\href{http://www.logilab.org/xmldiff/}{http://www.logilab.org/xmldiff/})
We could of course try to use perl/grep/awk to extract the 
information we need, but this is somewhat opposed to the motivation for using XML.


%-------------------------------------------------------------------
\section{Types of Test}

\subsection{Gauge Invarience}
We need a gauge invariance test subroutine in the Columbia code.  We can use the standalone UKQCD code to  produce gauge rotated configurations. 

   \href{http://www.ph.ed.ac.uk/ukqcd/collaboration/t3e_codes/transform_gauge/gaugetransform.html}{http://www.ph.ed.ac.uk/ukqcd/collaboration/t3e\_codes/transform\_gauge/gaugetransform.html}

These configurations could be read into the Columbia code.


%-------------------------------------------------------------------
\section{Parallel Random Number Generation}
Concerning the need to reproducability of results over a system of a given
size, no matter how that system is decomposed over a set of processors.
To achieve this, we must consider distributed and decomposition independent 
(Pseudo Random Number Generators) PRNGs.

Date: Tue, 11 Sep 2001 12:41:07 +0100 (BST)\\
From: "S.Booth" <spb@epcc.ed.ac.uk>\\
Reply-To: s.booth@epcc.ed.ac.uk\\
To: qcdoc-app@epcc.ed.ac.uk\\
Subject: Parallel random number generation

First of all some trivial notation
\begin{itemize}
 \item{$S$} - the state of a generator
 \item{$X$} - the output of a generator
 \item{$U$} - update transform (maps one state to the next)
 \item{$F$} - output function (maps state to output)
 \item{$P$} - period of generator
\end{itemize}
i.e.
\begin{equation}
 S_{n+1} = U.S_{n}
\end{equation}

\begin{equation}
 X_{n} = F.S_{n}
\end{equation}

\begin{equation}
  U^{P} \equiv I
\end{equation}


Assume we are generating random numbers in large batches (size N)
corresponding to a lattice worth of random numbers. (The
problem of accept/reject algorithms that consume different amounts of
random numbers on different sites is not too hard but I'll neglect it
here for clarity)

The easiest way of doing this in a decomposition independent way is to
store an RNG state at each site and use $U^N$ as the update transform.

\begin{equation}
 S_0   S_1   S_2  . . .     S_{N+1}
\end{equation}
goes to
\begin{equation}
 S_N   S_{N+1}   S_{N+2}  . . .     S_{2N-1}
\end{equation}

giving the same result as if a single generator had been run on a
sequential machine.

Vectorisable random number generators often used this trick, e.g for a
multiplicative  generator:

\begin{equation}
  X_{n+1}     = a . X_n  \mbox{mod} M
\end{equation}
implies
\begin{equation}
  X_{n+64}     = b . X_n \mbox{mod} M;  \hspace{5mm}  b = a^{64} \mbox{mod} M
\end{equation}
allowing the sequence to be vectorised with a vector length of 64
storing 64 words of state instead of 1.
(Linear congruential generators LCGs are very similar to this).
Even for a large offset $V a^V \mbox{mod} M$ can be calculated in $\log(V)$ time.


The same trick can also be used with other PRNG algorithms e.g. lagged
fibonacci. The difference here being that $U^N$ may be more expensive to
compute that U.  A lagged fibonacci update might be of the form
\begin{equation}
    X_n   =  X_{n-p} \pm  X_{n-q} \mbox{mod} M
\end{equation}
$U^N$ will need q multiplications and additions (mod M)
\begin{equation}
X_{n+N}= \sum_{i=1}^{q}    a_i .X_{n-i}  \mbox{mod} M
\end{equation}
(Note this cost is independent of the size of $N$ but the cost of
 calculating the constants $\{a_1, a_2 ... a_q\}$ is proportional to
 $\log(N)$, however you only need to do this once)

This additional cost is an artifact of how the original generator was
designed. The lagged fibonacci generators are just a special case of a
more general class of generator the Multiply recursive generator (MRG)
\begin{equation}
X_n  = \sum_{i=1}^q   a_i .X_{n-i} \mbox{mod} M
\end{equation}

It is just as easy (if not easier) to design just as good a
generator where $U^N$ is cheap to evaluate and U is the more expensive.
There is no such thing as a "good" random number generator. Generators
should really be anlysed in terms of their prospective use. However 
RNG designers usually use some fairly arbitrary criteria based on
short distance correlation in the sequance that are
thought to be valid for the majority of applications.

As far as a QCD simulation is concerned the important thing is the
statistical properties of $U_x$ $U_y$ $U_z$ $U_t$ and $U_s$ the operators that encode
the transformation the RNG state from one lattice site to the next and
between algorithmic iterations.
For a naive sequential implementation:
\begin{eqnarray}
  U_x &=& U \nonumber \\
  U_y &=& U^{Lx} \nonumber \\
  U_z &=& U_y^{Lx} \nonumber \\
  U_t &=& U_z^{Lz} \nonumber \\
  U_s &=& U_t^{Lt} \nonumber 
\end{eqnarray}
and we could investigate these properties with a varient of the
spectral test.

In the parallel implementation we can take
\begin{equation}
  U_s = U  
\end{equation}
and are free to choose $U_x$ $U_y$ $U_z$ $U_t$ for their statistical
properties rather than as a side effect of the order we loop through
the lattice.

The only problem with this approach is the increase in storage needed for the
generator state (q times the size of the lattice). This is only the
"working" storage, as the states at each site are all related by $U_{x,y,z,t}$
it is only necessary to checkpoint a single state the others can be
regenerated easily.


Its possible to trade off computation for storage in three ways.
\begin{enumerate}
\item We can get away with a smaller q if we use a prime modulus M rather
than a power of 2. a 32 bit power of 2 generator has maximum period $2^31.(2^q - 1)$ 
if M is prime the maximum period is $(M^q -1)$
Prime modulus at approx 31 bits won't be too expensive if we have
64-bit integer types available


\item Only store a single RNG state per processor. This is 
an alternative approach to getting decomposition independence where we
arrange for each lattice of random numbers generated to be the same as 
if they had been produced by a single processor. Each processor only
stores a single RNG state but applies a different transformation at
the end of each X, Y, Z, loop across the lattice. These transforms are
more expensive than the normal update transform ( roughly $q^2$
additions and multiplies for lagged fibonacci above) This is fine for
large local volumes but is probably a significant expense for our
target local volumes.
(On the T3E I did something like this but used a different
decomposition for the random numbers as for the physics
lattices. However that requires non local communications)

\item The option of having one generator per sub-cube of the lattice as
sugessted by bob would also work fine, this uses less storage than the
one generator / site approach but requires greater book-keepin. The
problem being that it constrains the possible lattice decompositions
to sub-cube boundaries. We could make the size of sub-cube a run-time
parameter so the same code can also run with 1 generator/site though
this increases the book-keeping overhead even futher.
\end{enumerate}

For HMC the RNG is not a very significant part of the runtime
so we have a lot of freedom about what to do. We could also design an
interface that could support any of the above approaches making it
easy to change later. 

At the moment I would suggest a 1 generator/site approach using a
prime modulus Multiply recursive generator (MRG) and 5 words of state
per site. (that's 5K bytes for a $4^4$ lattice) this generator would have
a period of $(P^5 - 1)$ with  $p = 2^31-1$  i.e. approx $2^155$

In many ways this is overkill but the RNG is such a small part of the
runtime a gold plated solution seems like a good idea to me.


\end{document}


