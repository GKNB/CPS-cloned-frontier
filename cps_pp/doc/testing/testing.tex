%------------------------------------------------------------------
% $Id: testing.tex,v 1.8 2004/08/18 11:57:35 zs Exp $
%------------------------------------------------------------------
%Anj: EPCC: e-mail: a.jackson@epcc.ed.ac.uk
%
% For best results, this latex file should be compiled using pdflatex.
% However it will also compile under normal latex, if you prefer.
%
% The conversion of this latex document to html is best done
% using the  "tth" program.
%
%------------------------------------------------------------------
\documentclass[12pt]{article}

% importing other useful packages:
\usepackage{times}
\usepackage{fullpage}
\usepackage{graphicx}
\usepackage{longtable}
\usepackage{tabularx}
% color for the pdf links:
\usepackage{color}
\definecolor{darkblue}{rgb}{0.0,0.0,0.5}
% for conditional latex source:
\usepackage{ifthen}
% pdftex specifications, these are only included if we are using pdflatex:
\providecommand{\pdfoutput}{0}
\ifthenelse{\pdfoutput = 0}{
% Not PDF:
\usepackage{html}
\newcommand{\hreff}[2]{\htmladdnormallink{#2}{#1}}
}{
% PDF: hyperref for pdf with full linkages:
\usepackage[
pagebackref,
hyperindex,
hyperfigures,
colorlinks,
linkcolor=darkblue,
citecolor=darkblue,
pagecolor=darkblue,
urlcolor=blue,
%bookmarksopen,
pdfpagemode=None,
%=UseOutlines,
pdfhighlight={/I},
pdftitle={Frameworks For Testing. $Revision: 1.8 $ - $Date: 2004/08/18 11:57:35 $.},
pdfauthor={A.N. Jackson, C. McNeile, Z. Sroczynski, \& S. Booth},
plainpages=false
]{hyperref}
}


% Code style commands:
\newcommand{\cls}[1]{{\bf #1}}            % Classes
\newcommand{\struct}[1]{{\bf #1}}         % Structs
\newcommand{\cde}[1]{{\tt #1}}            % Code fragments

% document style modifications:
\setlength{\parskip}{2.0mm}
\setlength{\parindent}{0mm}

% Questionbox commands:
\newcounter{quescount}
\setcounter{quescount}{0}
\newcommand{\quesbox}[2]{\begin{center}\refstepcounter{quescount}\fbox{\parbox{130mm}{
\label{#1}{\bf Q. \arabic{quescount}:} #2} } \end{center} }


% title information:
\title{Testing framework for the Columbia code.}
\author{A.N. Jackson, C. McNeile, , Z. Sroczynski,  \& S. Booth}
\date{\mbox{\small $$Revision: 1.8 $$  $$Date: 2004/08/18 11:57:35 $$}}

%------------------------------------------------------------------
\begin{document}

\maketitle

\tableofcontents
\newpage

%-------------------------------------------------------------------
\section{Motivation for Testing Framework}

We are ``growing'' a testing framework for the cps++ code.
The aim is to produce a basic regression testing framework to allow
``refactoring'' and physics checks on relevant parts of the operating
system.
The history of the testing framework is described in 
section~\ref{se:history}.


This testing framework is meant to be used for the CPS code on its own
without any additional libraries, such as 
\href{http://www.jlab.org/~edwards/qdp/}{QDP++}.
  In particular we
assume that none of the output is in XML format, hence we can not use the
\href{https://forge.nesc.ac.uk/projects/xmldiff/}{XMLDiff}
tool developed by EPCC for UKQCD.  There will be an additional
testing framework for the hybrid applications that use a mixture of
CPS and QDP++ libraries.

\subsection{Running the tests}

The tests are run using the makefile target.
\begin{verbatim}
make -f Makefile test
\end{verbatim}
%%
This produces output like
\begin{verbatim}
COMPILATION STATUS
f_hmd SUCCESS
f_hmd_dwfso SUCCESS
g_hb SUCCESS
g_hmd SUCCESS
s_spect SUCCESS
w_spect SUCCESS
f_wilson_eig SUCCESS
xi_spect_gsum SUCCESS
f_stag_pbp SUCCESS
t_asqtad_pion SUCCESS
TEST_CASE_STATUS
f_hmd    PASS
f_hmd_dwfso    FAIL
g_hb    FAIL
g_hmd    FAIL
s_spect    PASS
w_spect    PASS
f_wilson_eig  NO_CORRECT_OUTPUT
xi_spect_gsum  NO_CORRECT_OUTPUT
f_stag_pbp  NO_CORRECT_OUTPUT
t_asqtad_pion PASS
------------------------------
DISCLAIMER
Please also test this code on
a physical system before using in
production runs
------------------------------
\end{verbatim}

A perl script loops over the test cases, compiles the code,
and then runs it. The output is compared against
previous output and a PASS/FAIL tag is generated.
If there is no test data to compare against the
NO\_CORRECT\_OUTPUT tag is printed.

The latest version of the system first compiles all the 
applications and then runs them. This allows the user to
see which applications compile and which don't.


\subsection{Philosophy behind the testing framework}

Lattice QCD codes produce a lot of output. It is important that the 
output is automatically tested against older output, otherwise
a user could miss a problem.

Because of rounding errors, the output can not be tested against older
output by using a crude tool such as diff. We have taken the
philosophy that only an important subset of the data needs to be
tested. For example, a regression test may only need to check that the
pion correlator at a single timeslice is correct. If the pion
correlator at timeslice 2 is correct, then it is likely that all the
meson correlators are correct. This will trigger a PASS/FAIL
print out, but we would still keep a snapshot of all the output.

In this testing framework, we use perl scripts to
extract a subset of the correlators. The important
perl script that does the comparison is called
regressions/check\_data.pl. This compares the output from
the program to previous output stored in a special format.



\subsection{Adding a new test to the testing framework}

In the older testing framework all the files were 
combined together to form a single file of the form.

\begin{verbatim}
-------------------
a0.dat
--------------------
1 0 0 -1.4003954066e+00
1 1 0 -1.0403947310e+00
--------------------
a0_prime.dat
--------------------
1 0 0 2.2492961048e+00
1 1 0 -6.7672838306e-01
\end{verbatim}

The script regressions/check\_data.pl stores selected data
from the file in the format
%%%
\begin{center}
name:column:row\_from\_name:value
\end{center}
%%
In the above example the testing data for the w\_spect
test is:

\begin{verbatim}
a0.dat:2:3:-1.4003954061e+00
pion.dat:3:3:1.5681429497e+00
nucleon.dat:2:3:1.179528e+00
\end{verbatim}


\subsection{Structure of makefile target}

The test target in the Makefile looks like:
%%%%
\begin{verbatim}
TEST_DIR = tests

test: cps
        cd $(TEST_DIR) ; perl regression.pl
        cd $(TEST_DIR) ; sh regression.sh
\end{verbatim}

The perl script regression.pl writes the script 
regression.sh. The script regression.sh compiles the codes,
runs them and checks the output. This two step solution
was required for running on QCDSP.

\subsection{Testing framework code in CPS}

In the original design of CPS some of the physics
measurement code was buried a number of layers 
in a class hierarchy. To test code the output from 
the routines needs to close to the main program.
Hence some simple testing code has been placed 
in
\begin{verbatim}
src/util/testing_framework
\end{verbatim}

At the moment there is code to compute the local pion for staggered
fermions and to compare two arrays. We would expect that this 
would evolve into an API.

The code is used in the test ``t\_asqtad\_pion'' like:

\begin{verbatim}
  // known output
  Float pion_corr_good[] = { 7.563369 , 6.546032 , 5.474610 ,
  5.255596 } ;
  int good_size = sizeof(pion_corr_good) / sizeof(Float) ;

  // compute the time sliced pion correlator
  int time_size=GJP.Tnodes()*GJP.TnodeSites();
  GwilsonFasqtad lat;
  Float* pion_corr = (Float*)smalloc(time_size*sizeof(IFloat));
  staggered_local_pion(lat,cg_arg.mass,pion_corr,time_size) ;

  Float tol = 0.00001 ;
  compare_array_relative(pion_corr, pion_corr_good,tol,
                           time_size) ;
\end{verbatim}

Links to the some of 
\href{../examples}{the test codes }
are now
included in the documentation bundle.
This is the place to include a short explanation of 
what the test is meant to check.

\section{Things to do} 

\begin{itemize}


\item The idea of using the perl script regression.pl to write
      the shell script regression.sh is a bit ugly.
      As the name of the test also needs to be correlated with the 
      type of comparison test, it might be better to use a single
      python script. Each test has a number of things 
      associated with it, hence some kind of object oriented 
      solution would be useful. The OO features of perl are
      a bit weird.

\end{itemize}

%-------------------------------------------------------------------

\section{History of the testing framework} \label{se:history}

The original testing framework was written by Chris Miller
at Columbia. The testing framework was taken over and improved
by Andy Jackson (EPCC). 

Zbyszek Sroczynski and Craig McNeile are currently maintaining
the system.



\end{document}


