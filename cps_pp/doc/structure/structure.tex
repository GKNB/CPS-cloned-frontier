

%------------------------------------------------------------------
% $Id: structure.tex,v 1.3 2004-02-01 16:36:21 mcneile Exp $
%------------------------------------------------------------------
%Anj: EPCC: e-mail: a.jackson@epcc.ed.ac.uk
%
% For best results, this latex file should be compiled using pdflatex.
% However it will also compile under normal latex, if you prefer.
%
%------------------------------------------------------------------
\documentclass[12pt]{article}

% importing other useful packages:
\usepackage{times}
\usepackage{fullpage}
\usepackage{graphicx}
\usepackage{longtable}
\usepackage{tabularx}
% color for the pdf links:
\usepackage{color}
\definecolor{darkblue}{rgb}{0.0,0.0,0.5}
% for conditional latex source:
\usepackage{ifthen}
% pdftex specifications, these are only included if we are using pdflatex:
\providecommand{\pdfoutput}{0}
\ifthenelse{\pdfoutput = 0}{
% Not PDF:
\usepackage{html}
\newcommand{\hreff}[2]{\htmladdnormallink{#2}{#1}}
}{
% PDF: hyperref for pdf with full linkages:
\usepackage[
pagebackref,
hyperindex,
hyperfigures,
colorlinks,
linkcolor=darkblue,
citecolor=darkblue,
pagecolor=darkblue,
urlcolor=blue,
%bookmarksopen,
pdfpagemode=None,
%=UseOutlines,
pdfhighlight={/I},
pdftitle={CPS: Overall Code Structure. $Revision: 1.3 $ - $Date: 2004-02-01 16:36:21 $.},
pdfauthor={A.N. Jackson \& S. Booth},
plainpages=false
]{hyperref}
}

% CVS tag shorthand:
\newcommand{\ukqcdtag}[1]{{\small{UKQCD CVS Repository Tag: }{\bf{\tt{#1}}}}}

% Code style commands:
\newcommand{\cls}[1]{{\bf #1}}            % Classes
\newcommand{\struct}[1]{{\bf #1}}         % Structs
\newcommand{\cde}[1]{{\tt #1}}            % Code fragments

% document style modifications:
\setlength{\parskip}{2.0mm}
\setlength{\parindent}{0mm}

% Questionbox commands:
\newcounter{quescount}
\setcounter{quescount}{0}
\newcommand{\quesbox}[2]{\begin{center}\refstepcounter{quescount}\fbox{\parbox{130mm}{
\label{#1}{\bf Q. \arabic{quescount}:} #2} } \end{center} }


% title information:
\title{CPS: Overall Code Structure.}
\author{A.N. Jackson \& S. Booth}
\date{\mbox{\small $$Revision: 1.3 $$  $$Date: 2004-02-01 16:36:21 $$}}

%------------------------------------------------------------------
\begin{document}

\maketitle

\tableofcontents
\newpage

%-------------------------------------------------------------------
\section{File types and filename extensions}
\begin{tabular}{|l|llll|l|l|}
\hline
{\bf Type} & \multicolumn{4}{c|}{{\bf QCDSP}} & {\bf Elsewhere} & \multicolumn{1}{c|}{\bf GNU}\\
     & General & Tartan & T.I. & Sun &  & \\
\hline
C++ source & .C & - & - & .C & .C & .[cc$|$cxx$|$cpp$|$c++$|$C] \\
C/C++ header & .h & - & - & .h & .h & .h\\ 
Pre-processed file & .i & - & - & .i & .i & .i\\
Assembler source & .asm & - & - & .asm & .asm & .[s$|$S] \\
Object & - & .tof & .obj & .o & .o & .o \\
Archive/Library & - & .olb & .lib & .lib & .a & .a \\
Binary & - & .outtof & .out & .out & .out & - \\
Output Data & .dat & - & - & .dat & .dat & - \\
GNU build input files & - & - & - & - & .in & .in \\
\hline
\end{tabular}

Note that QCDSP also uses linker control files (.lcf), object list files
(.ctl), linker map files (.map) and processed assembler files (.lst) [???].  
See the \href{../qcdsp/toplevel.html#tth_chAp6}{QCDSP manual}
for more information on all the file types used on that platform.


%-------------------------------------------------------------------
\section{./phys/ - The Physics Code}
This is the top level directory.  It contains global configuration files (e.g.
config.h) and the makefiles.

\subsection{./phys/alg/ - Algorithms}
\label{phys:alg}
Physical simulation algorithms.

\subsection{./phys/doc/ - Documentation}
\label{phys:doc}
Documentation for the code, starting at
\href{../index.html}{./phys/doc/index.html}.

\subsection{./phys/lib/ - Library Files}
\label{phys:lib}
All the library archive files from all the other code directories are placed
here to be linked against.

\subsection{./phys/mem/ - Memory Routines}
\label{phys:mem}
Mainly p2v.C, which copies assembler routines into CRAM for the QCDSP machine.

\subsection{./phys/nga/ - Node Gate Array Routines}
\label{phys:nga}
The hardware interface routines.  The global sums and the circular buffer
routines are defined in here.  This directory also contains the MPI
implementation of the SCU layer.

\subsection{./phys/task/ - Task \& Scripting Routines}
\label{phys:task}
Mainly for QCDSP, these routines parse and execute scripts describing
sequences of QCD computations.

\subsection{./phys/tests/ - The Regression Testing Programs}
\label{phys:tests}
A set of programs for regression testing of the code suite.

\subsection{./phys/util/ - The Physics Utilies Library}
\label{phys:util}
This contains the core code, defining the data structures and the essential 
mathematical algorithms. e.g. the CG solver is defined in here, in context,
but called from ./phys/alg/.

%-------------------------------------------------------------------
\section{The GNU build system}
When not running on QCDSP, the CPS is built using standard GNU tools to make
the platform dependencies as simple to deal with as possible.  The tools are
autoconf (\href{http://www.gnu.org/manual/autoconf/}{http://www.gnu.org/manual/autoconf/}) and GNU make (gmake, \href{http://www.gnu.org/manual/make/}{http://www.gnu.org/manual/make/}). 

\subsection{autoconf: resolving platform dependancies and build options}
The core of this standard automatic configuration system is the developer's
configure script `configure.in' (in the root ./phys/ directory of the CPS).  
This defines which files are system/build dependent, and defines macro 
substitutions to produce versions of those files suitable for a particular
build.  The configure.in script cannot be run directly, so we need to
translate this file into a shell script that will run on as wide a range of
platforms as possible.  This is what autoconf does; it takes configure.in and
produces a shell script called `configure' which should run on any flavour of 
UNIX.  The user invokes the `./configure' command to
configure the build.

The build dependant files are:

\begin{tabular}{lll}
{\bf Input} & {\bf Output} & {\bf Description}\\
 Makefile.gnu.in & Makefile.gnu & The makefile for building the CPS library.
 \\
 Makefile.gnutests.in & Makefile.gnutests & The makefile for building the test
 suite. \\
 config.h.in & config.h & The global code-configuration header file. \\
 tests/regression.pl.in & tests/regression.pl & A perl script to generate
 shell scripts for running the test suite. \\
\end{tabular}

So, for example, config.h is produced from config.h.in, by performing
macrosubstitution of all the @MACRO\_NAME@ macros in that file.  The full list
of macros is:

\begin{tabular}{lll}
{\bf Macro} & {\bf Description} & {\bf Default}\\
@CC@ & The C compiler & ``gcc'' \\
@CFLAGS@ & Flags for the C compiler & ``-O'' \\
@NOT\_TESTING\_QCDSP@ & Is this not QCDSP & ``yes'' \\
@PARALLELDEF@ & Define the PARALLEL macro & ``-DPARALLEL=0'' \\
@MPILINK@ & To link with MPI use e.g. -lmpi &  ``'' \\
@PARALLELLINK@ & Link with parallel or serial libraries & ``SERIALLIB'' \\
@TESTING\_PARALLEL@ & Should the tests be run as parallel & ``no'' \\
@GSUMPRECISION@ & Floating-point precision to use for global sums & ``double''
\\
@LOCALPRECISION@ & Floating-point precision to use elsewhere & ``double'' \\
@topwd\_srcdir@ & Absolute location of the top-level source directory & - \\
\end{tabular}

The user can use the configure script to change the behaviour away from the
default.  Currently, the build options are:
\begin{itemize}
\item {\bf --enable-parallel-mpi} which turns the build into a parallel build using
the MPI implementation of the SCU.
\item {\bf --enable-double-prec=``no''} which forces the build to use floats
instead of doubles for all calculations except the global summations.
\end{itemize}
There are other standard ./configure options (use ``./configure --help'' to
see them) but these are not really used in the CPS.

Note that mny of the macros can be overriden using shell environment
variables.  For example, if you want to use the Sun C++ compiler with
debugging information switched on, i.e. CC -g , you should use (in bash):

\begin{verbatim}
export CC=CC
export CFLAGS=-g
./configure
\end{verbatim}

You should be warned, however, that the configure script stores a record of
the build options in a file called config.status or config.cache.  You may have
to remove this file to force the build behaviour to change.

There are a number of other special files that autoconf creates to store
information and to ensure that the ./configure system will run anywhere.  They
are config.cache, config.guess, config.log, config.status, config.sub,
install-sh, missing, mkinstalldirs, uname.  See the autoconf manual for
detailed information on what these files do.



\subsection{gmake: the GNU make program}
The GNU version of make is used simply because it has a little more
functionality that standard make programs.  In particular, the makefiles can
be (and are) recursive.

\section{Outline of the new code structure}

\subsection{A unified build system}

\subsection{Cross-platform portability}

\subsection{Cross-platform testing suite}

%-------------------------------------------------------------------
\end{document}

